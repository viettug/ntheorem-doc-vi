\section{Examples}\label{sec:examples}

The setting is as follows. 
\begin{itemize}
 \item For Theorems:
  \begin{verbatim}
   \theoremstyle{marginbreak}
   \theoremheaderfont{\normalfont\bfseries}\theorembodyfont{\slshape}
   \theoremsymbol{\ensuremath{\diamondsuit}}
   \theoremseparator{:}
   \newtheorem{Theorem}{Theorem}\end{verbatim}
 \item For Lemmas:
  \begin{verbatim}
   \theoremstyle{changebreak}
   \theoremsymbol{\ensuremath{\heartsuit}}
   \theoremindent0.5cm
   \theoremnumbering{greek}
   \newtheorem{Lemma}{Lemma}\end{verbatim}
 \item For Corollaries:
  \begin{verbatim}
   \theoremindent0cm
   \theoremsymbol{\ensuremath{\spadesuit}}
   \theoremnumbering{arabic}
   \newtheorem{Corollary}[Theorem]{Corollary}\end{verbatim}
 \item For Examples:
  \begin{verbatim}
   \theoremstyle{change}
   \theorembodyfont{\upshape}
   \theoremsymbol{\ensuremath{\ast}}
   \theoremseparator{}
   \newtheorem{Example}{Example}\end{verbatim}
 \item For Definitions:
  \begin{verbatim}
   \theoremstyle{plain}
   \theoremsymbol{\ensuremath{\clubsuit}}
   \theoremseparator{.}
   \newtheorem{Definition}{Definition}\end{verbatim}
 \item For Proofs:
  \begin{verbatim}
   \theoremheaderfont{\sc}\theorembodyfont{\upshape}
   \theoremstyle{nonumberplain}
   \theoremseparator{}
   \theoremsymbol{\rule{1ex}{1ex}}
   \newtheorem{Proof}{Proof}\end{verbatim}
\end{itemize}
Note, that parts of the setting are inherited. For instance, the
fonts are not reset before defining ``Lemma'', so the font setting 
of ``Theorem'' is used.

\begin{Example}[Simple one]
 The first example is just a text. 

 In the next examples, it is shown how an endmark is put at a
 displaymath, a single equation and both types of eqnarrays.
\end{Example}

\begin{Theorem}[Long Theorem]
 The examples are put into this theorem environment. 

The next example will not appear in the list of examples since
it is written as
\begin{quote}
 |\begin{Example*} ... \end{Example*}|
\end{quote}
\begin{Example*}[Ending with a displayed formula]
Look, the endmark is really at the bottom of the line:
\[ f^{(n)}(z) =
   \frac{n!}{2\pi i} \int \limits _{\partial D}
            \frac{f(\zeta)}{(\zeta-z)^{n+1}} d\zeta \]
\end{Example*}
At this point, we add an additional entry without number 
in the Example list:
\begin{verbatim}
\addtheoremline*{Example}{Extra Entry}\end{verbatim}
\addtheoremline*{Example}{Extra Entry}
\begin{Lemma}[Display with array]
Lemmata are indented and numbered with greek symbols.
Also for displayed arrays of this form, it looks good:
\begin{verbatim}
\[\begin{array}{l}
     a = \begin{array}[t]{l}
           first\ line \\
           second\ line
         \end{array}%
     \mbox{try to put this text in the lowest line}\end{array}\] \end{verbatim}
Just try to get this with the presented array structure ... without
using dirty tricks, you can position the outer array either [t], [c],
or [b], and you will not get the desired effect.
\[\begin{array}{l}
     a = \begin{array}[t]{l}
           first\ line \\
           second\ line
         \end{array} \mbox{try to put this text in the lowest line}
  \end{array}\]
\end{Lemma}
\begin{Lemma}[Equation]
For |equation|s, we decided to put the endmark after the equation
number, which is vertically centered.
Currently, we do not know, how to get the equation number centered and
the endmark at the bottom (one has to know the internal height of the
math material) ... If anyone knows, please inform us.
\begin{equation}
 \int_{\gamma} f(z)\, dz := \int_a^b f(\gamma (t)) \gamma'(t) \, dt
\end{equation}
\end{Lemma}

With the |break|-theoremstyles, if the environment is labeled and 
written as
\begin{quote}
|\begin{Lemma}[Breakstyle]\label{breakstyle}|
\end{quote}
\begin{Lemma}[Breakstyle]\label{breakstyle}
you see, there is a leading space \dots \\
If a percent (comment) (or an explicit |\ignorespaces|) is put directly 
after the label, e.g.
\begin{quote}
 |\begin{Lemma}[Breakstyle]\label{breakstyle}%|, 
\end{quote}
the space disappears.

From the predefined styles, this is exactly the case for the break-styles.
That's no bug, it's \LaTeX-immanent.

\noindent
The example goes on with an |eqnarray|:
\begin{eqnarray}
f(z) &=&
   \frac{1}{2\pi i}
   \int \limits_{\partial D} \frac{f(\zeta)}{\zeta-z} d\zeta \\
&= &
   \frac{1}{2\pi}
   \int \limits_0^{2\pi}
      f(z_0 + re^{it}) dt
\end{eqnarray}
\end{Lemma}

\begin{Proof}[of nothing]
\begin{eqnarray*}
f(z) &=&
   \frac{1}{2\pi i}
   \int \limits_{\partial D} \frac{f(\zeta)}{\zeta-z} d\zeta \\
&= &
   \frac{1}{2\pi}
   \int \limits_0^{2\pi}
      f(z_0 + re^{it}) dt
\end{eqnarray*}
\end{Proof}
That's it (the end of the Theorem).
\end{Theorem}


If there are some environments in the same thm-environment,
the last gets the endmark:
\begin{Definition}[With a list]
\begin{equation}
 \int_{\gamma} f(z)\, dz := \int_a^b f(\gamma (t)) \gamma'(t) \, dt
\end{equation}
\begin{itemize}
\item you've seen, how it works for text and
\item math environments,
\item and it works for lists.
\end{itemize}
\end{Definition}

\begin{Corollary}[Q.E.D.]
And here is a trivial corollary, which is ended by
|\qedsymbol{\textrm{q.e.d}}| and |\qed|.
\qedsymbol{q.e.d}\qed
\end{Corollary}

\begin{Example}
\[ f^{(n)}(z) =
   \frac{n!}{2\pi i} \int \limits _{\partial D}
            \frac{f(\zeta)}{(\zeta-z)^{n+1}} d\zeta \]
If there is some text after an environment, the endmark is put
after the text.
\end{Example}

The next one is done by the following sequence. Note, that 
|~\hfill~| is inserted to prevent \LaTeX\ from using its nested list 
management (a verbatim is also a trivlist),
i.e.\ this causes \LaTeX\ to start the |verbatim|-Part in a new line.
\begin{quote}
|\begin{Example}| \\
|~\hfill~| \\
|\begin{verbatim}| \\
|And, it also works for verbatim|\\
|... when the \end{verbatim} is in the|\\
|same line as the text ends. \end{verbatim}|\\
|                           ^| this space is important !!\\
|\end{Example}|
\end{quote}

\begin{Example}[Using |verbatim|]
~\hfill~
\begin{verbatim}
And, it also works for verbatim
... when the end{verbatim} is in the
same line as the text ends. \end{verbatim}
\end{Example}

There must be no empty line in the input before the |\end{theorem}|
(since then, the end mark is ignored) \\
\begin{quote}
|\begin{Theorem}| \\ 
|some text ... but no end mark| \\
| | \\
|\end{Theorem}|
\end{quote}

\begin{Theorem}\label{ex-empty-line}
some text ... but no end mark

\end{Theorem}


Now, there is a corollary which should appear with a different
name in the list of corollaries:
\begin{quote}
 |\begin{Corollary*}[title in text]\label{otherlabel}| \\
 |...|\\
 |\end{Corollary*}|
 |\addtheoremline{Corollary}{title in list}| \\
\end{quote}
\begin{Corollary*}[title in text]\label{otherlabel}\ignorespaces
\begin{center}
   It also works in the \\
   center \\
   environment.  
\end{center}
\end{Corollary*}
\addtheoremline{Corollary}{title in list}

\begin{Theorem}[Quote]
\begin{quote}
In quote environments, the text is normally indented from left 
and right by the same space. The endmark is not indented from the 
right margin, i.e., it is typeset to the right margin of the
surrounding text.
\end{quote}
\end{Theorem}

Here is an example for turning off the endmark automatics and
manual handling:

\begin{verbatim}
\begin{Theorem}[Manual End Mark]\label{somelabel}
a line of text with a manually set endmark \hfill\TheoremSymbol \\
some more text, but no automatic endmark set. \NoEndMark
\end{Theorem}\end{verbatim}

\begin{Theorem}[Manual End Mark]\label{somelabel}
a line of text with a manually set endmark \hfill\TheoremSymbol \\
some more text, but no automatic endmark set. \NoEndMark
\end{Theorem}
Also, one should note, that |\hfill| is inserted to set
the endmark at the right margin.

\begin{Example}[Quickie] It also works for short one's. 
\end{Example}

If you are tired of the greek numbers and the indentation for lemmata ... 
you can redefine it:
\theoremstyle{changebreak}
\theoremheaderfont{\normalfont\bfseries}\theorembodyfont{\slshape}
\theoremsymbol{\ensuremath{\heartsuit}}
\theoremsymbol{\ensuremath{\diamondsuit}}
\theoremseparator{:}
\theoremindent0cm
\theoremnumbering{arabic}
\renewtheorem{Lemma}{Lemma}
\begin{quote}
|\theoremstyle{changebreak}| \\
|\theoremheaderfont{\normalfont\bfseries}\theorembodyfont{\slshape}| \\
|\theoremsymbol{\ensuremath{\heartsuit}}|\\
|\theoremsymbol{\ensuremath{\diamondsuit}}|\\
|\theoremseparator{:}|\\
|\theoremindent0.5cm|\\
|\theoremnumbering{arabic}|\\
|\renewtheorem{Lemma}{Lemma}|
\end{quote}
\begin{Lemma}
  another lemma, with arabic numbering ... note that the numbering
  continues.
\end{Lemma}

the optional argument (i.e.\ the `theorem'-name) can be accessed by
|\|\meta{env}|name|. 

\begin{quote}
|\begin{Theorem}[somename]|\\
|Obviously, we are in Theorem~\Theoremname|.\\
|\end{Theorem}|
\end{quote}
\begin{Theorem}[somename]
Obviously, we are in Theorem~\Theoremname.
\end{Theorem}

This feature can e.g.\ be used for automatically generating
executable code and a commented solution sheet:
\begin{quote}
|\begin{exercise}[quicksort]| \\
\meta{the exercise text} \\
|\begin{verbatimwrite}{solutions/\exercisename.c}|\\
\meta{C-code} \\
|\end{verbatimwrite}|\\
|\verbatiminput{solutions/\exercisename.c}|\\
|\end{exercise}|
\end{quote}
This will write the C-code to a file |solutions/quicksort.c| and 
type it also on the solution sheet.

Now, we define an environment |KappaTheorem| which uses the same 
style parameters as Theorems and is numbered together with
Corollaries (Theorems are also numbered with Corollaries).
Note that we define a complex header text and a complex end mark.

|\theoremclass{Theorem}|\\
|\theoremsymbol{\ensuremath{a\atop b}}| \\
|\newtheorem{KappaTheorem}[Corollary]{\(\kappa\)-Theorem}|

\theoremclass{Theorem}
\theoremsymbol{\ensuremath{a\atop b}}
\newtheorem{KappaTheorem}[Theorem]{\(\kappa\)-Theorem}

\begin{KappaTheorem}[1st \(\kappa\)-Theorem]\label{kappatheorem1}
That's the first Kappa-Theorem. 
\end{KappaTheorem}

\subsection{Extended Referencing Features}\label{sec-ExtRef}[Section]

The standard |\label| command is extended by an optional argument
which is intended to contain the ``name'' of the structure which
is labeled, allowing more comfortable referencing; e.g., this
section has been started with

\begin{quote}
|\subsection*{Extended Referencing Features}%|\\
|\label{sec-ExtRef}[Section]|
\end{quote}

As already stated, for theorem-like environments the optional 
argument is filled in automatically, i.e., 
\begin{quote}
|\begin{Theorem}[Manual End Mark]\label{somelabel}|
\end{quote}
(cf.\ page~\pageref{somelabel}) is equivalent to
\begin{quote}
|\begin{Theorem}[Manual End Mark]\label{somelabel}[Theorem]|
\end{quote}

|\thref{|\meta{label}|}| additionally outputs the contents
of the optional argument which has been associated with \meta{label}:

\begin{quote}
|This is \thref{sec-ExtRef}| \\
|A theorem end mark has been set manually in \thref{somelabel}.| \\
|A center environment has been shown in \thref{otherlabel}.| \\
|The first Kappa-Theorem has been given in \thref{kappatheorem1}.|
\end{quote}

generates
\begin{quote}
This is \thref{sec-ExtRef}. \\
A theorem end mark has been set manually in \thref{somelabel}.
A center environment has been shown in \thref{otherlabel}. 
The first Kappa-Theorem has been given in \thref{kappatheorem1}.
\end{quote}

Here one must be careful that the handling of the optional 
argument is automated
only for environments defined by |\newtheorem|, i.e., \emph{not} 
for sectioning, equations, or enumerations. 

Calling |\thref{|\meta{label}|}| for a label which has been
set without an optional argument can result in different unintended
results: If \meta{label} is not inside a theorem-like environment, an
error message is obtained, otherwise the type of the surrounding
theorem-like environment is output, e.g., calling |\thref{label}| 
then results in ``Theorem~\meta{number}''!
Additionally, currently there is no support for multiple references 
such as ``see Theorems~5 and~7'' (this would require plural-forms
for different languages and handling of |\ref|-lists, probably
splitting into different sublists for different environments)\footnote{If 
someone is interested in programming this, please contact us; it 
seems to be algorithmically easy, but tedious.}.

\subsection{List of Theorems and Friends}

Note, that we put the following lists into the |quote|-environment
to emphazise them from the surrounding text. So the lists
are indented slightly at the margin.

With 
\begin{quote}
|\addtotheoremfile{Added into all theorem lists}|,
\end{quote}
in every list, an additional line of text would be inserted.
But it isn't actually done in this documentation since we want
to use different list formats.

Only for the list of Examples, this one is added: 
\begin{quote}
 |\addtotheoremfile[Example]{Only concerning Example lists}|
\end{quote}
\addtotheoremfile[Example]{Only concerning Example lists}

With
\begin{quote}
 |\theoremlisttype{all}| \\
 |\listtheorems{Lemma}|, 
\end{quote}
all lemmas are listed:
\begin{quote}
 \theoremlisttype{all}
 \listtheorems{Lemma}
\end{quote}

From the examples, only those are listed which have an optional name: 
\begin{quote}
 |\theoremlisttype{opt}| \\
 |\listtheorems{Example}|
\end{quote}
leads to
\begin{quote}
 \theoremlisttype{opt}
 \listtheorems{Example}
\end{quote}
One should note the line \emph{Only concerning example lists}, which
was added by the |\addtotheoremfile|-statement above.

For the next list, another layout, using the |tabular|-environment, 
is defined:
\begin{verbatim}
  \newtheoremlisttype{tab}%
    {\begin{tabular*}{\linewidth}{@{}lrl@{\extracolsep{\fill}}r@{}}}%
    {##1&##2&##3&##4\\}%
    {\end{tabular*}}\end{verbatim}
Thus, by saying
\begin{quote}
 |\theoremlisttype{tab}|\\
 |\listtheorems{Theorem,Lemma},|
\end{quote}
theorems and lemmata are listed:
\begin{quote}
  \DeleteShortVerb{\|}
    \newtheoremlisttype{tab}%
    {\begin{tabular*}{\linewidth}{@{}lrl@{\extracolsep{\fill}}r@{}}}%
    {##1&##2&##3&##4\\}%
    {\end{tabular*}}
   \theoremlisttype{tab}
   \listtheorems{Theorem,Lemma}
  \MakeShortVerb{\|}
\end{quote}

\LaTeX-lists can also be used to format the theoremlist.
The input
\begin{verbatim}
  \newtheoremlisttype{list}%
    {\begin{trivlist}\item}
    {\item[##2 ##1:]\ ##3\dotfill ##4}%
    {\end{trivlist}}
  \theoremlisttype{list}
  \listtheorems{Corollary}\end{verbatim}
leads to%
\begin{quote}%
\DeleteShortVerb{\|}%
  \newtheoremlisttype{list}%
    {\begin{trivlist}}%
    {\item[##2 ##1:]\ ##3\dotfill ##4}%
    {\end{trivlist}}%
  \theoremlisttype{list}\listtheorems{Corollary}%
  \MakeShortVerb{\|}%
\end{quote}%
In this example, after the item, \verb*!\ ! is used instead of 
\verb*! !, because in the latter case, |\dotfill| will produce an 
error if the optional argument (|##3|) 
is missing.

\endinput
