%% \section{Examples}
\section{\texorpdfstring{Ví dụ}{Vi du}}

\label{sec:examples}

%% The setting is as follows. 
Các thiết lập được dùng như sau đây.
Chú ý rằng, các thiết lập tuân theo nguyên tắc thừa kế. Ví dụ,
thiết lập |font| không được khởi tạo về trạng thái bình thường khi ta
khai báo môi trường |Lemma|, vì thế |Lemma| sẽ thừa kế các thiết lập
của môi trường |Theorem|.

\medskip
Một số ví dụ có dùng lệnh |\color| được cung cấp bởi gói |xcolor.sty|.

\medskip 
%% \begin{itemize}
%%  \item For Theorems:
%% \item 
Định lý:
\begin{command}
  \theoremstyle{marginbreak}
  \theoremheaderfont{\normalfont\bfseries}\theorembodyfont{\slshape}
  \theoremsymbol{\ensuremath{\diamondsuit}}
  \theoremseparator{:}
  \newtheorem{Theorem}{Theorem}\end{command}
%%  \item For Lemmas:
%% \item 
Bổ đề:
\begin{command}
  \theoremstyle{changebreak}
  \theoremsymbol{\ensuremath{\heartsuit}}
  \theoremindent0.5cm
  \theoremnumbering{greek}
  \newtheorem{Lemma}{Lemma}\end{command}
%%  \item For Corollaries:
%% \item
 Hệ quả:
\begin{command}
  \theoremindent0cm
  \theoremsymbol{\ensuremath{\spadesuit}}
  \theoremnumbering{arabic}
  \newtheorem{Corollary}[Theorem]{Corollary}
\end{command}
%%  \item For Examples:
%% \item 
Ví dụ:
\begin{command}
  \theoremstyle{change}
  \theorembodyfont{\upshape}
  \theoremsymbol{\ensuremath{\ast}}
  \theoremseparator{}
  \newtheorem{Example}{Example}\end{command}
%%  \item For Definitions:
%% \item
 Định nghĩa
\begin{command}
  \theoremstyle{plain}
  \theoremsymbol{\ensuremath{\clubsuit}}
  \theoremseparator{.}
  \newtheorem{Definition}{Definition}\end{command}
%%  \item For Proofs:
Chứng minh:
\begin{command}
  \theoremheaderfont{\sc}\theorembodyfont{\upshape}
  \theoremstyle{nonumberplain}
  \theoremseparator{}
  \theoremsymbol{\rule{1ex}{1ex}}
  \newtheorem{Proof}{Proof}
\end{command}
%% \end{itemize}
%% Note, that parts of the setting are inherited. For instance, the
%% fonts are not reset before defining ``Lemma'', so the font setting 
%% of ``Theorem'' is used.

%% \begin{thmbox}
{\color{-red!75!green!50}
\begin{Example}[Ví dụ đơn giản]
%%  The first example is just a text. 
Một chiều đi trên con đường này, Hoa điệp vàng trải dưới chân tôi,\ldots 

%%  In the next examples, it is shown how an endmark is put at a
%%  displaymath, a single equation and both types of eqnarrays.
	Các ví dụ tiếp theo minh họa cho việc đặt các dấu kết thúc
	ở các biểu thức toán, phương trình đơn lẻ và các dãy phương trình.
\end{Example}
%% \end{thmbox}
}

%% \begin{thmbox}
{%\color{red!60}
\begin{Theorem}[Định lý dài]
\label{thm:verylong}
%% The examples are put into this theorem environment. 
Ví dụ về môi trường |Theorem|, lồng bên trong nó là
các môi trường khác như |Example|, |Lemma|,\ldots Khi đang
ở ngay trong môi trường |Theorem|, chữ có màu đen; khi ở trong
các môi trường sâu hơn, màu của môi trường sẽ đổi khác để dễ phân biệt.

\medskip
%% The next example will not appear in the list of examples since
%% it is written as
Ví dụ tiếp theo sẽ không xuất hiện trong danh sách |Example|,
vì nó được dùng với dạng sao:
\begin{command}
  \begin{Example*}
    ...
  \end{Example*}
\end{command}
{\color{red!60}
\begin{Example*}%[Ending with a displayed formula]
[kết thúc với biểu thức toán]
%% Look, the endmark is really at the bottom of the line:
Hãy để ý vị trí dấu kết thúc ở dưới dòng biểu diễn công thức
\[ f^{(n)}(z) =
   \frac{n!}{2\pi i} \int \limits _{\partial D}
            \frac{f(\zeta)}{(\zeta-z)^{n+1}} d\zeta \]
\end{Example*}}
%% At this point, we add an additional entry without number 
%% in the Example list:
Bây giờ, ta sẽ ghi vài thông tin bổ sung vào danh sách |THM| (|Example|);
thông tin được ghi sẽ không gồm chỉ số |THM|:
\begin{command}
  \addtheoremline*{Example}{Extra Entry}
\end{command}
\addtheoremline*{Example}{Extra Entry}
{\color{red!75!green}
\begin{Lemma}%[Display with array]
[Biểu thức trong mảng (dãy) phương trình]
%% Lemmata are indented and numbered with greek symbols.
%% Also for displayed arrays of this form, it looks good:
Bổ để (|Lemma|) được thụt đầu dòng và đánh chỉ số với chữ số Hy Lạp.

\medskip
%% Ngoài ra, các biểu thức như sau đây trông có vẻ như ý:
Xét ví dụ sau đây, trông có vẻ tốt:
\begin{command}
 \[\begin{array}{l}
      a = \begin{array}[t]{l}
            first\ line \\
            second\ line
         \end{array}%
     \mbox{\color{green}{try to put this text in the lowest line}}
   \end{array}
  \]
\end{command}
%% Just try to get this with the presented array structure ... without
%% using dirty tricks, you can position the outer array either [t], [c],
%% or [b], and you will not get the desired effect.
Nhưng biểu diễn như trên mà không sử dụng các mẹo đặc biệt nào
(thử để lựa chọn các vị trí |[t]|, |[c]|, |[b]|),
kết quả có thể không như ý:
\[\begin{array}{l}
   a = \begin{array}[t]{l}
         first\ line \\
         second\ line
       \end{array}%
   \mbox{\color{green}{try to put this text in the lowest line}}
  \end{array}\]
\end{Lemma}}
{\color{green!75!red}
\begin{Lemma}[Phương trình]
%% For |equation|s, we decided to put the endmark after the equation
%% number, which is vertically centered.
%% Currently, we do not know, how to get the equation number centered and
%% the endmark at the bottom (one has to know the internal height of the
%% math material) ... If anyone knows, please inform us.
Với các phương trình có đánh số, dấu kết thúc sẽ được đặt sau chỉ số
phương trình, và được canh giữa theo chiều đứng. Hiện tại, chưa có thuật
toán để bố trí chỉ số phương trình canh giữa (theo chiều đứng),
còn dấu kết thúc lại được đặt vào cuối phương trình --- việc này đỏi hòi
phải biết chiều cao của nội dung phương trình. Nếu ai đó biết,
vui lòng báo cho tác giả của |ntheorem|.
\begin{equation}
 \int_{\gamma} f(z)\, dz := \int_a^b f(\gamma (t)) \gamma'(t) \, dt
\end{equation}
\end{Lemma}}

%% With the |break|-theoremstyles, if the environment is labeled and 
%% written as
Với kiểu |break|, nếu sử dụng việc đánh nhãn như sau
\begin{command}
  \begin{Lemma}[Breakstyle]\label{breakstyle}
\end{command}
{\color{red!75!green!50!blue}
\begin{Lemma}[Breakstyle]\label{breakstyle}
%% you see, there is a leading space \dots \\
%% If a percent (comment) (or an explicit |\ignorespaces|) is put directly 
%% after the label, e.g.
thì như bạn thấy, xuất hiện các khoảng trắng thừa \ldots \\
Nếu dấu phần trăm (chú thích) được đặt ngay sau lệnh tạo nhãn
(hoặc dùng lệnh |\ignorespaces|), ví dụ
\begin{command}
  \begin{Lemma}[Breakstyle]\label{breakstyle}% 
\end{command}
%% the space disappears.
thì khoảng trắng thừa sẽ mất.

%% From the predefined styles, this is exactly the case for the break-styles.
%% That's no bug, it's \LaTeX-immanent.
\medskip
Với các kiểu đã được định nghĩa bởi |ntheorem|, điều này chỉ xảy ra
chỉ với kiểu |break|. Đây không phải là lỗi, mà chính là cách xử lý của \LaTeX{}.

\medskip
%% The example goes on with an |eqnarray|:
Ví dụ với |eqnarray|:
\begin{eqnarray}
f(z) &=&
   \frac{1}{2\pi i}
   \int \limits_{\partial D} \frac{f(\zeta)}{\zeta-z} d\zeta \\
&= &
   \frac{1}{2\pi}
   \int \limits_0^{2\pi}
      f(z_0 + re^{it}) dt
\end{eqnarray}
\end{Lemma}}

{\color{green}
\begin{Proof}[of nothing]
\begin{eqnarray*}
f(z) &=&
   \frac{1}{2\pi i}
   \int \limits_{\partial D} \frac{f(\zeta)}{\zeta-z} d\zeta \\
&= &
   \frac{1}{2\pi}
   \int \limits_0^{2\pi}
      f(z_0 + re^{it}) dt
\end{eqnarray*}
\end{Proof}}
%% That's it (the end of the Theorem).
Đến đây là kết thúc Định lý~\ref{thm:verylong}.
\end{Theorem}
%% \end{thmbox}
}

%% If there are some environments in the same thm-environment,
%% the last gets the endmark:
Nếu có nhiều môi trường cùng bên trong một môi trường |THM|,
môi trường cuối cùng mới có dấu kết thúc
\begin{Definition}[với danh sách]
\begin{equation}
 \int_{\gamma} f(z)\, dz := \int_a^b f(\gamma (t)) \gamma'(t) \, dt
\end{equation}
\begin{itemize}
%% \item you've seen, how it works for text and
%% \item math environments,
%% \item and it works for lists.
\item như vậy, bạn đã thấy dấu kết thúc làm việc thế nào với văn bản
\item với biểu thức toán
\item và với danh sách.
\end{itemize}
\end{Definition}

\begin{Corollary}[Q.E.D.]
%% And here is a trivial corollary, which is ended by
Đây là hệ quả tầm thường, kết thúc bởi
|\qedsymbol{\textrm{q.e.d}}|  và |\qed|.
\qedsymbol{q.e.d}\qed
\end{Corollary}

\begin{Example}
\[ f^{(n)}(z) =
   \frac{n!}{2\pi i} \int \limits _{\partial D}
            \frac{f(\zeta)}{(\zeta-z)^{n+1}} d\zeta \]
%% If there is some text after an environment, the endmark is put
%% after the text.
Nếu có văn bản theo sau môi trường, dấu kết thúc sẽ đặt sau văn bản đó.
\end{Example}

%% The next one is done by the following sequence. Note, that 
%% |~\hfill~| is inserted to prevent \LaTeX\ from using its nested list 
%% management (a verbatim is also a trivlist),
%% i.e.\ this causes \LaTeX\ to start the |verbatim|-Part in a new line.
\medskip
Ví dụ tiếp theo được cho bởi mã nguồn sau đây. Chú ý rằng,
lệnh |~\hfill~| được chèn vào để ngăn cản \LaTeX{} quản lý danh sách lồng nhau
theo cách của chính \LaTeX{} (môi trường |verbatim| là danh sách |\trivlist|).
Có nghĩa là, việc này sẽ khiến \LaTeX{} bắt đầu môi trường |verbatim|
với một dòng mới.
\begin{command}
  \begin{Example}
  ~\hfill~
  \begin{verbatim}
  And, it also works for verbatim
  ... when the \end{verbatim} is in the
  same line as the text ends. \end{verbatim}
                                ^ this space is important !!
  \end{Example}
\end{command}

\begin{Example}[dùng `verbatim']
~\hfill~
\begin{verbatim}
And, it also works for verbatim
... when the end{verbatim} is in the
same line as the text ends. \end{verbatim}
\end{Example}

%% There must be no empty line in the input before the |\end{theorem}|
%% (since then, the end mark is ignored) \\
Không được chừa dòng trắng trước |\end{theorem}|, bởi làm thế
sẽ bỏ qua dấu kết thúc.
\begin{command}
  \begin{Theorem}
   some text ... but no end mark
   
  \end{Theorem}
\end{command}

\begin{Theorem}\label{ex-empty-line}
some text ... but no end mark

\end{Theorem}


%% Now, there is a corollary which should appear with a different
%% name in the list of corollaries:
Bây giờ là hệ quả sẽ xuất hiện với tên riêng hơi khác
trong danh sách |Corollary|:
\begin{command}
  \begin{Corollary*}[title in text]\label{otherlabel}
  ...
  \end{Corollary*}
  \addtheoremline{Corollary}{title in list}
\end{command}
\begin{Corollary*}[title in text]%
\label{otherlabel}\ignorespaces
\begin{center}
   Thể hiện trong \\
   môi trường \\
   canh giữa.  
\end{center}
\end{Corollary*}
\addtheoremline{Corollary}{title in list}

\begin{Theorem}[trích dẫn]
\begin{quote}
%% In quote environments, the text is normally indented from left 
%% and right by the same space. The endmark is not indented from the 
%% right margin, i.e., it is typeset to the right margin of the
%% surrounding text.
Trong môi trường trích dẫn |quote|, nội dung thường được thụt vào ở tất
cả các dòng ở cả bên trái và bên phải. Tuy nhiên, dấu kết thúc sẽ không
vẫn được đặt đúng vào lề bên phải.
\end{quote}
\end{Theorem}

%% Here is an example for turning off the endmark automatics and
%% manual handling:
Dưới đây là ví dụ về việc tắt/bật việc đặt dấu kết thúc tự động.

\begin{command}
  \begin{Theorem}[Manual End Mark]\label{somelabel}
  a line of text with a manually set endmark \hfill\TheoremSymbol\\
  some more text, but no automatic endmark set. \NoEndMark
  \end{Theorem}
\end{command}

\begin{Theorem}[Manual End Mark]\label{somelabel}
a line of text with a manually set endmark \hfill\TheoremSymbol \\
some more text, but no automatic endmark set. \NoEndMark
\end{Theorem}

%% Also, one should note, that |\hfill| is inserted to set
%% the endmark at the right margin.
Để ý rằng, lệnh |\hfill| được chèn vào trước dấu kết thúc
để đặt dấu đó vào lề bên phải.

\begin{Example}[nhanh hơn]
%%  It also works for short one's. 
Dấu kết thúc được tự động đặt trở lại...
\end{Example}

%% If you are tired of the greek numbers and the indentation for lemmata ... 
%% you can redefine it:
Nếu bạn không thích dùng chỉ số là chữ số Hy Lạp và cách thụt đầu dòng cho
|Lemma|, bạn có thể định nghĩa lại:
\theoremstyle{changebreak}
\theoremheaderfont{\normalfont\bfseries}\theorembodyfont{\slshape}
\theoremsymbol{\ensuremath{\heartsuit}}
\theoremsymbol{\ensuremath{\diamondsuit}}
\theoremseparator{:}
\theoremindent0cm
\theoremnumbering{arabic}
\renewtheorem{Lemma}{Lemma}
\begin{command}
  \theoremstyle{changebreak}
  \theoremheaderfont{\normalfont\bfseries}\theorembodyfont{\slshape}
  \theoremsymbol{\ensuremath{\heartsuit}}
  \theoremsymbol{\ensuremath{\diamondsuit}}
  \theoremseparator{:}
  \theoremindent0.5cm
  \theoremnumbering{arabic}
  \renewtheorem{Lemma}{Lemma}
\end{command}
\begin{Lemma}
%%   another lemma, with arabic numbering ... note that the numbering
%%   continues.
	Bây giờ là bổ đề khác, với cách đánh số dùng chữ số Ả Rập. Chú ý rằng
	chỉ số vẫn tiếp tục tăng.
\end{Lemma}

%% the optional argument (i.e.\ the `theorem'-name) can be accessed by
%% |\|\meta{env}|name|. 
Tham số bổ sung xác định tên riêng của |THM| có thể lấy được nhờ 
chẳng hạn |\Theoremname| hoặc |Examplename|,\ldots --- điều này chỉ
có thể làm bên trong môi trường |THM| hiện tại mà thôi.

\begin{command}
  \begin{Theorem}[\color{red}{some name}\normalcolor]
  Obviously, we are in Theorem~\Theoremname.
  \end{Theorem}
\end{command}
\begin{Theorem}[\color{red}{some name}\normalcolor]
Obviously, we are in Theorem~\Theoremname.
\end{Theorem}

%% This feature can e.g.\ be used for automatically generating
%% executable code and a commented solution sheet:
Tính năng có thể dùng, chẳng hạn để sinh ra tự động các mã
cho môi trường chú thích, |verbatim|,\ldots:
\begin{command}
  \begin{exercise}[quicksort]
   `\meta{the exercise text}'
  \begin{verbatimwrite}{solutions/\exercisename.c}
   `\meta{C-code}'
  \end{verbatimwrite}
  \verbatiminput{solutions/\exercisename.c}
  \end{exercise}
\end{command}
%% This will write the C-code to a file |solutions/quicksort.c| and 
%% type it also on the solution sheet.
Đoạn mã trên sẽ viết mã \meta{C-code} vào tập tin |solutions/quicksort.c|,
và sau đó nạp vào nhờ lệnh |\verbatiminput|.

%% Now, we define an environment |KappaTheorem| which uses the same 
%% style parameters as Theorems and is numbered together with
%% Corollaries (Theorems are also numbered with Corollaries).
%% Note that we define a complex header text and a complex end mark.
\medskip
Bây giờ, ta định nghĩa môi trường |KappaTheorem| sử dụng cùng kiểu
như môi trường |Theorem|, được đánh số tiếp theo các Hệ quả (|Corollary|)
(các |Theorem| cũng được đánh số theo |Corollary|). Để ý rằng,
ta sẽ đưa ra phần |header| và phần dấu kết thúc khá phức tạp.

\begin{command}
  \theoremclass{Theorem}
  \theoremsymbol{\ensuremath{a\atop b}}
  \newtheorem{KappaTheorem}[Theorem]{\(\kappa\)-Theorem}
\end{command}

\theoremclass{Theorem}
\theoremsymbol{\ensuremath{a\atop b}}
\newtheorem{KappaTheorem}[Theorem]{\(\kappa\)-Theorem}

\begin{KappaTheorem}[1st \(\kappa\)-Theorem]\label{kappatheorem1}
Đây là định lý |kappa| đầu tiên. 
\end{KappaTheorem}

% =====================================================================

%% \subsection{Extended Referencing Features}
\subsection{\texorpdfstring{Tham khảo mở rộng}{Tham khao mo rong}}%
\label{sec-ExtRef}[Mục]

%% The standard |\label| command is extended by an optional argument
%% which is intended to contain the ``name'' of the structure which
%% is labeled, allowing more comfortable referencing; e.g., this
%% section has been started with
Lệnh |\label| chuẩn được mở rộng: bây giờ nó có thể nhận thêm tham số
bổ sung dùng để xác định tên cấu trúc dùng để đánh nhãn -- nhờ đó
việc tham khảo chéo được linh hoạt hơn.
Ví dụ, mục này được đánh nhãn như sau:
\begin{command}
  \subsection*{....}%
  \label{sec-ExtRef}[Mu.c]
\end{command}

%% As already stated, for theorem-like environments the optional 
%% argument is filled in automatically, i.e., 
Như đã nói, với môi trường |THM|, phần tham số bổ sung của |\label|
sẽ được tự động thêm vào. Do đó, 
\begin{command}
|\begin{Theorem}[Manual End Mark]\label{somelabel}|
\end{command}
%% (cf.\ page~\pageref{somelabel}) is equivalent to
tương đương với (xem \vpageref{somelabel})
\begin{command}
|\begin{Theorem}[Manual End Mark]\label{somelabel}[Theorem]|
\end{command}

%% |\thref{|\meta{label}|}| additionally outputs the contents
%% of the optional argument which has been associated with \meta{label}:
Lệnh |\thref{|\meta{label}|}| sẽ sinh ra thông tin bổ sung, cho biết
tên |THM| tương ứng với tham số bổ sung của |\label|. Ví dụ:

\begin{command}
  This is \thref{sec-ExtRef}
  ... \thref{somelabel}
  ... \thref{otherlabel}
  ... \thref{kappatheorem1}
\end{command}
sẽ cho ta
%% generates
\begin{quote}
This is \thref{sec-ExtRef}.
  ... \thref{somelabel}
  ... \thref{otherlabel}
  ... \thref{kappatheorem1}
\end{quote}

%% Here one must be careful that the handling of the optional 
%% argument is automated
%% only for environments defined by |\newtheorem|, i.e., \emph{not} 
%% for sectioning, equations, or enumerations. 
Phải cẩn thận: việc xử lý tham số bổ sung
được tiến hành tự động chỉ cho các môi trường được định nghĩa nhờ |\newtheorem|,
nghĩa là sẽ không có sự quản lý cho các Mục (|\section|),
phương trình (|equation|) hay danh sách (|enumerate|).

%% Calling |\thref{|\meta{label}|}| for a label which has been
%% set without an optional argument can result in different unintended
%% results: If \meta{label} is not inside a theorem-like environment, an
%% error message is obtained, otherwise the type of the surrounding
%% theorem-like environment is output, e.g., calling |\thref{label}| 
%% then results in ``Theorem~\meta{number}''!
Việc gọi |\thref{|\meta{label}|}| cho các nhãn chưa được thiết lập với
phần tham số bổ sung sẽ sinh ra các kết quả khó tưởng tượng: nếu \meta{label}
không ở bên trong môi trường |THM|, lỗi sẽ sinh ra; ngược lại, tên |THM|
hiện tại sẽ được dùng, ví dụ việc gọi |\thref{xxxlabel}|
sẽ sinh ra chẳng hạn ``Định lý~\meta{number}''!

\medskip
Chú ý rằng không có hỗ trợ cho các tham khảo chéo phức, như ``xem Định lý~5 và~7''.%
\footnote{This would require plural-forms
for different languages and handling of \texttt{\bslash ref}-lists, probably
splitting into different sublists for different environments. If 
someone is interested in programming this, please contact us;
\color{red}{it seems to be algorithmically easy, but tedious.}}

%% Additionally, currently there is no support for multiple references 
%% such as ``see Theorems~5 and~7'' (this would require plural-forms
%% for different languages and handling of |\ref|-lists, probably
%% splitting into different sublists for different environments)\footnote{If 
%% someone is interested in programming this, please contact us; it 
%% seems to be algorithmically easy, but tedious.}.

%% \subsection{List of Theorems and Friends}
\subsection{\texorpdfstring{Danh sách THM}{Danh sach THM}}

%% Note, that we put the following lists into the |quote|-environment
%% to emphazise them from the surrounding text. So the lists
%% are indented slightly at the margin.
%% Chú ý rằng, ở đây các danh sách được bố trí trong các môi trường đặc biệt
%% để nhấn mạnh kết quả, còn thực tế, chúng sẽ được thể hiện với lề trái
%% khác ở đây một chút!.

%% With 
\medskip
Với
\begin{command}
  \addtotheoremfile{Added into all theorem lists}
\end{command}
%% in every list, an additional line of text would be inserted.
%% But it isn't actually done in this documentation since we want
%% to use different list formats.
thì ở mọi danh sách, dòng ``|Added into all theorem lists|''
sẽ được chèn vào. Tuy nhiên, trong tài liệu này, ta không làm như thế,
vì ta dùng định dạng khác của danh dách |THM|.

%% Only for the list of Examples, this one is added: 
\medskip Để chèn chỉ danh sách các ví dụ (|Example|), có thể làm chẳng hạn
\begin{command}
  \addtotheoremfile[Example]{%
    \color{blue}{Only concerning Example lists}}
\end{command}
\addtotheoremfile[Example]{\color{blue}{Only concerning Example lists}}

%% With
\medskip
Với
\begin{command}
  \theoremlisttype{all}
  \listtheorems{Lemma} 
\end{command}
%% all lemmas are listed:
mọi bổ đề sẽ được liệt kê

\bigskip
{\color{red!75!green!50!blue}
 \theoremlisttype{all}
 \listtheorems{Lemma}}
\bigskip

%% From the examples, only those are listed which have an optional name: 
Như ta thấy trong kết quả trên, xuất hiện dòng chỉ toàn dấu chấm.
Ta sử dụng kiểu danh sách |opt| để liệt kê chỉ các bổ đề có tên riêng:
\begin{command}
  \theoremlisttype{opt}
 \listtheorems{Example}
\end{command}
%% leads to
cho ta kết quả

\bigskip
{\color{red!75!green!50!blue}
 \theoremlisttype{opt}
 \listtheorems{Example}
}

\bigskip
%% One should note the line \emph{Only concerning example lists}, which
%% was added by the |\addtotheoremfile|-statement above.
Như ta thấy, dòng chữ ``\color{blue}{Only concerning Example lists}\normalcolor''
xuất hiện ở cuối danh sách --- ta đã thêm dòng này nhờ |\addtotheoremfile|
ở ví dụ trên.

\medskip
%% For the next list, another layout, using the |tabular|-environment, 
%% is defined:
Bây giờ ta định nghĩa kiểu danh sách mới, sử dụng môi trường |tabular|:
\begin{command}
  \newtheoremlisttype{tab}%
   {\begin{tabular*}{\linewidth}%
    {@{}lrl@{\extracolsep{\fill}}r@{}}}%
    {##1&##2&##3&##4\\}%
   {\end{tabular*}}
\end{command}
%% Thus, by saying
Sử dụng kiểu mới |tab| như sau:
\begin{command}
  \theoremlisttype{tab}
  \listtheorems{Theorem,Lemma}
\end{command}
%% theorems and lemmata are listed:
Các định lý và bổ đề sẽ được liệt kê theo kiểu mới:

\bigskip
{\color{red!75!green!50!blue}
%% \begin{quote}
  \DeleteShortVerb{\|}
    \newtheoremlisttype{tab}%
    {\begin{tabular*}{\linewidth}{@{}lrl@{\extracolsep{\fill}}r@{}}}%
    {##1&##2&##3&##4\\}%
    {\end{tabular*}}
   \theoremlisttype{tab}
   \listtheorems{Theorem,Lemma}
  \MakeShortVerb{\|}%
%% \end{quote}%
}

\vskip-12pt

%% \LaTeX-lists can also be used to format the theoremlist.
%% The input
Cũng có thể sử dụng môi trường tạo danh sách của \LaTeX{}
để tạo kiểu mới như sau đây:
\begin{command}
  \newtheoremlisttype{list}%
    {\begin{trivlist}\item}
    {\item[##2 ##1:]\ ##3\dotfill ##4}%
    {\end{trivlist}}
  \theoremlisttype{list}
  \listtheorems{Corollary}
\end{command}
%% leads to%
Kết quả với kiểu mới bây giờ là

{\color{red!75!green!50!blue}
%% \begin{quote}%
\DeleteShortVerb{\|}%
  \newtheoremlisttype{list}%
    {\begin{trivlist}\item}%
    {\item[##2 ##1:]\ ##3\dotfill ##4}%
    {\end{trivlist}}%
  \theoremlisttype{list}
  \listtheorems{Lemma}%
  \MakeShortVerb{\|}%
%% \end{quote}%
}

%% In this example, after the item, \verb*!\ ! is used instead of 
%% \verb*! !, because in the latter case, |\dotfill| will produce an 
%% error if the optional argument (|##3|) 
%% is missing.
Trong ví dụ này, sau mỗi phần tử |\item| của danh sách,
lệnh \verb*#\ # được dùng thay cho \verb*# #, vì trong trường hợp sau,
lệnh |\dotfill| sẽ báo lỗi nếu phần tham số bổ sung |##3| không có.	

\endinput
