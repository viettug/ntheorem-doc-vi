\documentclass[12pt]{article}

\usepackage{ntheorem}
\usepackage[utf8x]{vietnam}

\let\:\overline

\theoremstyle{marginbreak}
\newtheorem{thm}{Định lý}

\theoremstyle{plain}
\newtheorem{thm2}[thm]{Mệnh đề}

\theoremstyle{changebreak}
\newtheorem{thm3}[thm]{Định lý}

\begin{document}

\begin{thm}[Ánh xạ Weingarten]
Nếu $k_1$, $k_2$ là các giá trị riêng của ánh xạ Weingarten của $X$,
thì $kk_1$, $kk_2$ là các giá trị riêng của ánh xạ Weingarten của $\:X$.
\end{thm}

\begin{thm2}[Ánh xạ Weingarten]
Nếu $k_1$, $k_2$ là các giá trị riêng của ánh xạ Weingarten của $X$,
thì $kk_1$, $kk_2$ là các giá trị riêng của ánh xạ Weingarten của $\:X$.
\end{thm2}

\end{document}
