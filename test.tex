\documentclass[12pt]{article}

% =====================================================================

\usepackage[utf8x]{vietnam}

\usepackage{amsmath}
%\usepackage{varioref}
\usepackage[thref,thmmarks,standard,amsmath,hyperref]{ntheorem}
\usepackage[]{hyperref}

% =====================================================================

\makeatletter
%% \def\thref#1{%
%%    \csname r@#1@type\endcsname
%%    \expandafter\ifx\csname r@#1@type\endcsname\None
%%      \PackageWarning{\basename}{thref: Reference Type of `#1' on page
%%       \thepage \space undefined}\G@refundefinedtrue
%%      \else\csname r@#1@type\endcsname~\fi%
%%    \csname r@#1@type\endcsname  
%% %% \expandafter\@setref\csname r@#1\endcsname\@firstoftwo{#1}     
%%    \ref{#1}}
\makeatother

% =====================================================================

\let\:\overline
\def\cW{\mathrm{W}}

% =====================================================================

\theoremstyle{marginbreak}
\newtheorem{thm}{Định lý}

\theoremstyle{plain}
\newtheorem{thm2}[thm]{Mệnh đề}

\theoremstyle{changebreak}
\newtheorem{thm3}[thm]{Định lý}

% =====================================================================

\parindent0pt

\begin{document}

\section{Mục đầu tiên}
\label{sec:test}

\begin{thm}[Ánh xạ Weingarten]
Nếu $k_1$, $k_2$ là các giá trị riêng của ánh xạ Weingarten của $X$,
thì $kk_1$, $kk_2$ là các giá trị riêng của ánh xạ Weingarten của $\:X$.
Đây là \thmname.
\end{thm}

\begin{thm2}[Ánh xạ Weingarten]
\label{test}
Nếu $k_1$, $k_2$ là các giá trị riêng của ánh xạ Weingarten của $X$,
thì $kk_1$, $kk_2$ là các giá trị riêng của ánh xạ Weingarten của $\:X$.
\end{thm2}

\begin{Proof}
Thật vậy, nếu $v$ là một véc tơ riêng ứng với giá trị riêng $k_1$ của $\cW$,
thì $\cW(v)=k_1v$. Sử dụng (\ref{eq:W.W}), ta có
$\:\cW(T^{-1}v) = k\cdot T^{-1}(k_1v)=kk_1\cdot T^{-1}v$.
Vì $w=T^{-1}v\not=0$, ta có $w$ là véc tơ riêng của $\cW$ ứng với giá trị riêng $kk_1$.
\end{Proof}

\section{Mục tiếp theo}

\begin{thm}[Nhị thức Newtown]
Newtown đã tìm ra được
\begin{equation}\label{eq:W.W}
(a+b)^n=\sum_{k=0}^n{n\choose k}a^{n-k}b^k
\end{equation}
\end{thm}

\section{Kiểm tra}

Tham khảo mở rộng: Xem \thref{test} (cho bởi \verb#\thref{test}#)

\bigskip
Tham khảo thường: Xem \ref{sec:test} (mục), \ref{eq:W.W} (phương trình)

\bigskip
Danh sách định lý:

\listtheorems{thm,thm2}
\end{document}
