%% \section{Problems and Questions}
\section{\texorpdfstring{Vấn đề thường gặp}{Van de thuong gap}}

%% \subsection{Known Limitations}
\subsection{\texorpdfstring{Giới hạn của gói}{Gioi han cua goi} ntheorem}


\begin{itemize}
 \item
%%   Since |ntheorem.sty| uses the |.aux| file for storing
%%   information about the positions of endmarks, \LaTeX\ must be
%%   run twice for correctly setting the endmarks.
  |ntheorem| dùng tập tin |.aux| để lưu thông tin về dấu kết thúc,
  do đó, cần biên dịch tài liệu ít nhất hai lần để các dấu
  kết thúc được đặt đúng vị trí.
 \item
%%   Since |ntheorem.sty| uses the |.aux| file for storing
%%   information about lists in the |.thm| file, a minimum of
%%   two runs is needed. If theorems move in any of these
%%   runs up to five runs can be needed to generate correct lists.
	Cũng do sử dụng tập tin |.aux| để lưu thông tin về danh sách
	trong tập tin |.thm|, cần thêm ít nhất hai lần biên dịch nữa.
	Việc di chuyển các |THM| có thể cần tới 5 lần biên dịch
	để có danh sách |THM| đúng đắn.
 \item
%%   Since we need to expand the optional argument of theorems
%%   in various ways for the lists, we decided to copy the text verbatim
%%   into the |.thm| file. Thus, if you use things like |\thesection|
%%   etc., the list won't show the correct text. Therefore you shouldn't
%%   use any command that needs to be expanded.
	Gói xử lý phần tên riêng của |THM| (tham số bổ sung khi gọi môi trường)
	theo vài cách khác nhau trong danh sách, do đó, đã sao chép nguyên xi,
	không triển khai (|expand|) phần tham số bổ sung đó vào tập tin |.thm|.
	Hệ quả là, nếu bạn dùng chẳng hạn lệnh |\thesection| bên trong phần tên
	riêng thì kết quả sẽ không như ý. Bạn không nên dùng bất kỳ lệnh gì
	trong phần tên riêng |THM|!
 \item
%%   In nested environments ending at the same time,
%%   only the endmark for the inner environment is set, as the following
%%   example shows:
	Nếu các |THM| lồng nhau kết thúc ở cùng thời điểm,
	|ntheorem| chỉ lập dấu kết thúc cho môi trường sâu nhất (|level| cao nhất)
	Chẳng hạn
\begin{example}
  \begin{Lemma}
   Some text.
   \begin{Proof}
     The Proof
   \end{Proof}
  \end{Lemma}
\end{example}
%%  yields to
sẽ đưa tới kết quả sau

\begin{thmbox}
 \begin{Lemma}
   Some text.
   \begin{Proof} The Proof \end{Proof}
 \end{Lemma}
\end{thmbox}

%%  You can handle this by specifying something invisible after the end
%%  of the inner theorem. Then the endmark for the outer theorem is
%%  set in the next line:
Bạn có thể vượt qua nhược điểm này, bằng cách thêm một nội dung ẩn
sau khi kết thúc |THM| bên trong, khi đó, |THM| bên ngoài sẽ có dấu
kết thúc như ý. Quan sát ví dụ sau đây:
\begin{example}
  \begin{Lemma}
   Some text.
   \begin{Proof}
   	The Proof
   \end{Proof}~
  \end{Lemma}
\end{example}
%%  yields to
kết quả là

\begin{thmbox}
 \begin{Lemma}
  Some text.
  \begin{Proof} The Proof \end{Proof}~
 \end{Lemma}
\end{thmbox}

 \item
%%  Document option |fleqn| is problematic: |fleqn| handles
%%   equations not by |$$| but by lists (check what happens for
	Sử dụng tùy chọn |fleqn| khi gọi lớp tài liệu có thể sinh ra
	rắc rối. Lý do là tuỳ chọn |fleqn| điều khiển các phương trình
	không phải bởi |$$| mà bởi danh sách (thử kiểm tra xem điều
	gì xảy ra nếu bạn dùng
\begin{example}
  \begin{theorem} \[ displaymath \] \end{theorem}
\end{example}
%%   in standard \LaTeX:
%%   The displaymath is \emph{not} set in an own line).
	trong \LaTeX{} chuẩn: nội dung |displaymath| sẽ không được đặt ở dòng riêng.
%%   Also, for long formulas, the equation number and the endmark are
%%   smashed into the formula at the right text margin.
	Cũng như vậy, với công thức dài, chỉ số phương trình và dấu kết thúc
	có thể sẽ gần công thức hơn so với bình thường.
 \item
%%  Naturally, |ntheorem.sty| will not work correctly in
%%   combination with other styles which change the handling
%%   of
	Một cách tự nhiên, |ntheorem| không làm việc cùng với các kiểu (gói)
	liên quan đến
  \begin{enumerate}
%%    \item theorem-like environments, or 
%%    \item environments concerned with the handling of endmarks, e.g.
%%      |\[...\]|, |eqnarray|, etc.
	\item xử lý môi trường |THM| tựa định lý,
	\item xử lý dấu kết thúc (ví dụ |\[...\]|, |eqnarray|,\ldots)
  \end{enumerate}
 \item
%%  |ntheorem.sty| is compatible with Frank Mittelbach's
%%   |theorem.sty|, which is the most widespread style for setting
%%   theorems.
	|ntheorem| không tương thích với gói |theorem| của Frank Mittelbach,
	là một gói dùng để biểu diễn |THM| rất phổ biến.

%%   It cannot be used \emph{with} |theorem.sty|, but it can be
%%   used instead of it.
	Gói |ntheorem| không thể dùng chung với gói |theorem|,
	nhưng có thể dùng thay cho gói |theorem|.
\end{itemize}

%% \subsection{Known ``Bugs'' and Problems}
\subsection{\texorpdfstring{Các BUG đã biết}{Cac BUG da biet}}


\begin{itemize}
%%  \item Ending a theorem \emph{directly} after the text, e.g.
	\item Khi kết thúc môi trường |THM| ngay sau nội dung,
	dấu kết thúc sẽ bị bỏ qua. Ví dụ
\begin{example}
  \begin{Lemma} Lemma\end{Lemma}
\end{example}
%%  suppresses the endmark:
sẽ sinh ra

\begin{thmbox}
  \begin{Lemma} Lemma\end{Lemma}
\end{thmbox}

%%   Therefore a space or a newline should be inserted 
%%   before |\end{...}|.
	Vì vậy, ít nhất một khoảng trắng hoặc dấu ngắt dòng
	cần phải có trước |\end{...}|.
	Ngoài ra, trước |\end{...}| không được là dòng trắng.
%%  \item With theoremstyle break, if the linebreak would cause
%%   ugly linebreaking in the following text, it is suppressed.
\item
	Với kiểu |THM| |break|, nếu việc ngắt dòng sau |header| làm
	cho phần nội dung tiếp theo ``xấu xí'' (theo cách hiểu của \LaTeX{}),
	thì việc ngắt dòng đó sẽ bị bỏ qua.
\end{itemize}

%% \subsection{Open Questions}
\subsection{\texorpdfstring{Câu hỏi mở}{Cau hoi mo}}

Các câu hỏi này liên quan chủ yếu đến việc phát triển
gói |ntheorem| --- nghĩa là không có ích lắm với người dùng bình thường.
%% If someone has a solution to one of those questions, please inform us.
%% (You can be sure to be mentioned in the Acknowledgements.)
Nếu ai đó có câu trả lời cho một trong các câu hỏi dưới đây,
vui lòng thông báo cho tác giả gói |ntheorem|; tác giả của câu
trả lời sẽ có mặt trong |Acknowledgements|.

\begin{itemize}
\item
%% For |equation|s, we decided to put the endmark after the equation
%% number, which is vertically centered.
%% Currently, we do not know, how to get the equation number centered and
%% the endmark at the bottom (one has to know the internal height of the
%% math material).
Với các phương trình (biểu diễn nhờ môi trường |equation|), dấu kết thúc
sẽ được đặt sau chỉ số phương trình (được canh giữa theo chiều đứng).
Hiện tại, vẫn chưa có thuật toán để canh giữa chỉ số phương trình
đồng thời đặt dấu kết thúc ở bên dưới (việc này đòi hỏi phải biết
chiều cao của nội dung phương trình)
\item
%% The placement of endmarks is mainly based on a check whether \LaTeX\
%% is in an ordinary text line when encountering an end-of-environment.
%% This question is \emph{partially} answered by |\ifhmode|: In a text
%% line, \LaTeX\ is always in |\hmode|. 
Thuật toán đặt dấu kết thúc dựa trên kết quả của việc kiểm tra xem liệu phần nội
dung bình thường đã hết chưa (khi gặp kết thúc môi trường |\end{...}|).
Việc kiểm tra này có câu trả lời \emph{một phần} (|partially|)
nhờ |\ifhmode|: trong một dòng, \LaTeX{} luôn ở chế độ |\hmode|.
Nhưng sau các biểu thức toán, \LaTeX{} cũng ở chế độ |\hmode|. Vì thế,
%% But, after an displaymath, \LaTeX\ is also in |\hmode|. Thus,
%% additionally |\lastskip| is checked: after a displaymath, 
%% |\lastskip|=0 holds.
phải kiểm tra thêm về |\lastkip|: sau biểu thức toán thì |\lastskip=0|.
%% In most situations, when text has been written into a line, 
%% |\lastskip| $\neq$ 0. But, this does not hold, if the source code
%% is of the following form: |...text\label{bla}|: then, |\lastskip|=0.
%% In those situations, the endmark is suppressed. \\
%% ?? How can it be detected whether \LaTeX\ has just ended a displaymath?
Trong hầu hết trường hợp, khi nội dung vừa ghi xong vào một dòng, thì |\lastskip| $\neq0$.
Nhưng điều này không luôn chắc chắn: nếu mã nguồn có dạng
|...text\label{bla}|, thì (sau đó) |\lastskip=0|. Và khi đó, dấu kết thúc bị bỏ qua.

\medskip
Câu hỏi đặt ra là, làm thế nào để xác định thời điểm \LaTeX{} vừa kết thúc
việc biểu diễn công thức toán?
\item 
%% The above problem with the label: The break style enforces a linebreak
%% by |\hfill\penalty-8000| after the |\trivlist|-item. Thus, \TeX\
%% gets back into the horizontal mode. The label places a ``whatsit''
%% somewhere ... and, it seems that the ``whatsit'' makes \TeX\ think
%% that there is a line of text. 
Trong vấn đề trên về nhãn: kiểu |break| sẽ gắng ngắt dòng sau phần |header|
bằng cách dùng |\hfill\penalty-8000| đằng sau |item| của |\trivlist|.
Vì vậy, \TeX{} sẽ chuyển vào chế độ |horizontal|.
The label places a ``whatsit''
somewhere ... and, it seems that the ``whatsit'' makes \TeX\ think
that there is a line of text.\footnote{\texttt{kyanh:} ``whatsit'', đó là cái dzì vậy?}
\end{itemize}
%% If someone has a solution to one of those questions, please inform us.
%% (You can be sure to be mentioned in the Acknowledgements.)

\endinput
