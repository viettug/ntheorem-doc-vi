\section{Problems and Questions}
\subsection{Known Limitations}

\begin{itemize}
 \item Since |ntheorem.sty| uses the |.aux| file for storing
  information about the positions of endmarks, \LaTeX\ must be
  run twice for correctly setting the endmarks.
 \item Since |ntheorem.sty| uses the |.aux| file for storing
  information about lists in the |.thm| file, a minimum of
  two runs is needed. If theorems move in any of these
  runs up to five runs can be needed to generate correct lists.
 \item Since we need to expand the optional argument of theorems
  in various ways for the lists, we decided to copy the text verbatim
  into the |.thm| file. Thus, if you use things like |\thesection|
  etc., the list won't show the correct text. Therefore you shouldn't
  use any command that needs to be expanded.
 \item In nested environments ending at the same time,
  only the endmark for the inner environment is set, as the following
  example shows:
 \begin{verbatim}
  \begin{Lemma}
   Some text.
   \begin{Proof} The Proof \end{Proof}
  \end{Lemma}\end{verbatim}
 yields to
 \begin{Lemma}
  Some text.
  \begin{Proof} The Proof \end{Proof}
 \end{Lemma}

 You can handle this by specifying something invisible after the end
 of the inner theorem. Then the endmark for the outer theorem is
 set in the next line:
 \begin{verbatim}
  \begin{Lemma}
   Some text.
   \begin{Proof} The Proof \end{Proof}~
  \end{Lemma}\end{verbatim}
 yields to
 \begin{Lemma}
  Some text.
  \begin{Proof} The Proof \end{Proof}~
 \end{Lemma}

 \item Document option |fleqn| is problematic: |fleqn| handles
  equations not by |$$| but by lists (check what happens for
  \begin{quote}
  |\begin{theorem} \[ displaymath \] \end{theorem}| 
  \end{quote}
  in standard \LaTeX:
  The displaymath is \emph{not} set in an own line).
  Also, for long formulas, the equation number and the endmark are
  smashed into the formula at the right text margin.
 \item Naturally, |ntheorem.sty| will not work correctly in
  combination with other styles which change the handling
  of
  \begin{enumerate}
   \item theorem-like environments, or 
   \item environments concerned with the handling of endmarks, e.g.
     |\[...\]|, |eqnarray|, etc.
  \end{enumerate}
 \item |ntheorem.sty| is compatible with Frank Mittelbach's
  |theorem.sty|, which is the most widespread style for setting
  theorems.

  It cannot be used \emph{with} |theorem.sty|, but it can be
  used instead of it.
\end{itemize}

\subsection{Known ``Bugs'' and Problems}

\begin{itemize}
 \item Ending a theorem \emph{directly} after the text, e.g.
 \begin{quote}
 |\begin{Theorem} text\end{Theorem}| 
 \end{quote}
 suppresses the endmark:
 \begin{Theorem} text\end{Theorem}
  Therefore a space or a newline should be inserted 
  before |\end{...}|.
 \item With theoremstyle break, if the linebreak would cause
  ugly linebreaking in the following text, it is suppressed.
\end{itemize}

\subsection{Open Questions}
\begin{itemize}
\item
For |equation|s, we decided to put the endmark after the equation
number, which is vertically centered.
Currently, we do not know, how to get the equation number centered and
the endmark at the bottom (one has to know the internal height of the
math material).
\item
The placement of endmarks is mainly based on a check whether \LaTeX\
is in an ordinary text line when encountering an end-of-environment.
This question is \emph{partially} answered by |\ifhmode|: In a text
line, \LaTeX\ is always in |\hmode|. 
But, after an displaymath, \LaTeX\ is also in |\hmode|. Thus,
additionally |\lastskip| is checked: after a displaymath, 
|\lastskip|=0 holds.
In most situations, when text has been written into a line, 
|\lastskip| $\neq$ 0. But, this does not hold, if the source code
is of the following form: |...text\label{bla}|: then, |\lastskip|=0.
In those situations, the endmark is suppressed. \\
?? How can it be detected whether \LaTeX\ has just ended a displaymath?
\item 
The above problem with the label: The break style enforces a linebreak
by |\hfill\penalty-8000| after the |\trivlist|-item. Thus, \TeX\
gets back into the horizontal mode. The label places a ``whatsit''
somewhere ... and, it seems that the ``whatsit'' makes \TeX\ think
that there is a line of text. 
\end{itemize}
If someone has a solution to one of those questions, please inform us.
(You can be sure to be mentioned in the Acknowledgements.)

\endinput
