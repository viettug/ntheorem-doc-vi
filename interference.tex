\section{Possible Interferences}
Since |ntheorem| reimplements the handling of theorem-environments
completely, it is incompatible with every package also concerning
those macros.

Additionally, the |thmmarks| algorithm for placing endmarks 
requires modifications of several environments (cf.\ Section 
\ref{sec:code}).
Thus, environments which are reimplemented or additionally defined
by document options or styles are not covered by the endmark 
algorithm of |ntheorem.sty|.

The |[thref]| option changes the |\label| command and the treatment
of labels when reading the |.aux| file. Thus it is potentially
incompatible with all packages also changing |\label| (or
|\newlabel|). Compatibility with babel's |\newlabel| isa
achieved if babel is loaded before ntheorem.

\subsection{Interfering Document Options.}

|ntheorem.sty| also copes with the usual document options 
|leqno| and |fleqn|\footnote{although for \texttt{fleqn} and 
  long formulas
  reaching to the right margin, equation numbers and endmarks can
  be smashed over the formula since \texttt{fleqn} does not use
  \texttt{\bslash eqno} for controlling the setting of the equation
  number.}.
If one of those options is used in the |\documentclass|
declaration, it is automatically recognized by the |thmmarks| part
of |ntheorem.sty|.

If one of those options is not used in |\documentclass|, but
with |amsmath| (see next section), it must not be specified 
for |ntheorem|, since all |amsmath| environments detect this option 
by themselves.

\subsection{Combination with amslatex.}\label{sec:amslatex}

|ntheorem.sty| interferes with |amsmath.sty| and |amsthm.sty|.

Note, that the LaTeX amstex package |amstex.sty| (\LaTeX2.09) is
obsolete and you should use |amsmath| and |amstext| for
\LaTeXe\ instead.  Up to |ntheorem-1.18|, it is compatible with 
|amsmath-1.x|. Since |ntheorem-1.19|, it is (hopefully) compatible 
with |amsmath-2.x|.

We would be happy if someone knowing and using |amsmath| would
join the development and maintenance of this style.

\subsubsection{amsmath}

Compatibility with amsmath (end marks for math environments, and 
handling of labels in math environments) is provided in the option
|[amsmath]|, (i.e., if |\usepackage{amsmath}| is used then
\begin{itemize}
\item |\usepackage[thmmarks]{ntheorem}| must be completed to \\
|\usepackage[amsmath,thmmarks]{ntheorem}|), and also
\item |\usepackage[thref]{ntheorem}| must be completed to \\
|\usepackage[amsmath,thref]{ntheorem}|).
\end{itemize}
Note, that |amsmath| has to be loaded \emph{before} |ntheorem| 
since the definitions have to be overwritten.

\subsubsection{amsthm}

|amsthm.sty| conflicts with the definition of theorem
layouts in |theorem.sty|, some features of |amsthm.sty|
have been incorporated into option |[amsthm]| which has
to be used \emph{instead of} |\usepackage{amsthm}|.

The Option provides theoremstyles |plain|, |definition|, and 
|remark|, and a |proof| environment as in |amsthm.sty|. 

The |\newtheorem*| command is defined even without this
option. Note that |\newtheorem*| always switches to the
nonumbered version of the current theoremstyle which
thus must be defined.

The command |\newtheoremstyle| is not taken over from 
|amsthm.sty|. Also, |\swapnumbers| is not implemented.
Here, the user has to express his definitions by the 
|\newtheoremstyle| command provided by |ntheorem.sty|,
including the use of |\theoremheaderfont| and |\theorembodyfont|.
The options |[amsthm]| and |[standard]| are in conflict
since they both define an environment |proof|.

Thus, we recommend not to use
|amsthm|, since the features for defining theorem-like
environments in |ntheorem.sty|---following 
|theorem.sty|---seem to be more intuitive and user-friendly.

\subsection{Babel}\label{sec:babel}

The |[thref]| option interferes with the |babel| package, thus in 
case that |babel| is used, |ntheorem| has to be loaded \emph{after} 
|babel|.

\subsection{Hyperref}\label{sec:hyperref}

Since |hyperref| redefines the \LaTeX\ |\contentsline|-command, it breaks
with |ntheorem| below version 1.17. Since version 1.17, the option 
|[hyperref]| makes |ntheorem| work with |hyperref|.
Theoremlists will then get linked list.

WARNING: The definition and redefinition of Theorem List Layouts
(see Section~\ref{sec:listtypes}) isn't yet working with
the |hyperref|-package. 

\endinput
