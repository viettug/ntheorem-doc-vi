%\section{The User-Interface}
\section{Sử dụng gói}

% =====================================================================

%\subsection{How to include the package}
\subsection{Nạp gói}

%The package |ntheorem.sty| is included by
Gói |ntheorem.sty| có thể nạp như sau
\begin{command}
  \usepackage[`\meta{options}']{ntheorem}
\end{command}
%% where the optional parameter \meta{options} selects predefined
%% configurations and special requirements. 
với \meta{options} là danh sách các tuỳ chọn và các yêu cầu đặc biệt.

%% The following \meta{options} are available by now, concerning three
%% independent issues:
\medskip
Các tuỳ chọn được cho nhờ \meta{options} như sau: % liên quan đến ba vấn đề độc lập:
\begin{description}
%% \item[Predefining environments:] (see Section~\ref{sec:standard})
%%   With [standard] and [noconfig], it can be chosen, if and what
%%   file is used for activating a (user-defined) standard set of 
%%   theorem environments.
\item\DescribeOptions{standard,noconfig}\option{standard} \option{noconfig}\\
	xem Mục~\vref{sec:standard}. Với một
	trong hai tùy chọn |standard| và |noconfig|,
	bạn có thể lựa chọn việc sử dụng hoặc không tập hợp
	các môi trường |THM| đã được định nghĩa sẵn.
%% \item[Compatibility with amsthm:] option [amsthm] provides 
%%        compatibility with the theorem-layout
%%       commands of the |amsthm|-package (see Section~\ref{sec:amslatex}).
\item\DescribeOption{amsthm}\option{amsthm}\\
	tùy chọn |amsthm| khi được dùng sẽ bảo đảm tính tương thích với các môi trường
	|THM| cung cấp bởi gói |amsthm|. Xem Mục~\vref{sec:amslatex}.
%\item[Activation of endmarks:]
\item\DescribeOption{thmmarks}\option{thmmarks}\\
%% [thmmarks] enables the automatical placement of endmarks 
%%     (see \ref{sec:general}); when using the |amsmath|-package,
%%     [thmmarks] must be complemented by [amsmath] 
%%     (see Section~\ref{sec:amslatex}).
	tuỳ chọn |thmmarks| đồng ý để gói |ntheorem.sty| tự động đặt dấu kết thúc
	(|endmarks|) (xem Mục~\ref{sec:general}); khi dùng với gói |amsthm|,
	tùy chọn |thmmarks| phải được dùng kèm với tuỳ chọn |amsmath|.
	Xem thêm ở Mục~\ref{sec:amslatex}.
%% \item[Activation of extended reference features:]
%% [thref] enables the extended reference features 
%%     (see Section~\ref{sec-ExtRef}); when using the |amsmath|-package,
%%     [thref] must be complemented by [amsmath] 
%%     (see Section~\ref{sec:amslatex}).
\item\DescribeOption{thref}\option{thmref}\\
	tuỳ chọn |thref| cho phép mở rộng khả năng tham khảo chéo. Xem Mục~\vref{sec-ExtRef};
	khi dùng với gói |amsthm|, tuỳ chọn này phải đi kèm với tuỳ chọn |amsmath|.
	Xem thêm ở Mục~\ref{sec:amslatex}.
%% \item[Compatibility with hyperref:] option [hyperref] provides
%%       compability with the |hyperref|-package 
%%       (see section~\ref{sec:hyperref}).
\item\DescribeOption{hyperref}\option{hyperref}\\
	tuỳ chọn |hyperref| bảo đảm tương thích với gói |hyperref|.
	Xem Mục~\vref{sec:hyperref}.
\end{description}

Dưới đây là một ví dụ:
\begin{example}
  \usepackage{hyperref}
  \usepackage[hyperref,thmmarks,noconfig]{ntheorem}
\end{example}
Với cách nạp gói như trên, bạn sẽ phải tự định nghĩa các môi trường |THM|,
các dấu kết thúc sẽ được định vị tự động. Vì ta dùng gói |hyperref|,
ta phải bảo đảm tính tương thích nhờ tuỳ chọn |hyperref|.

% =====================================================================

%\subsection{Defining New Theorem Sets}
\subsection{Định nghĩa THM mới}

\DescribeMacro{\newtheorem}\macro{newtheorem}\\
%% The syntax and semantics is exactly the same as in standard
%% \LaTeX{}: the command |\newtheorem| defines a new ``theorem set'' 
%% or ``theorem-like structure''.
%% Two required arguments name the new environment set and give the 
%% text to be typeset with each instance of the new ``set'', while
%% an optional argument determines how the ``set'' is enumerated:
Cú pháp của lệnh hoàn toàn giống như của lệnh chuẩn |\newtheorem|.
Lệnh sẽ định nghĩa một |THM| mới. Có hai tham số bắt buộc là tên
của môi trường và tên của |THM|. Tham số bổ sung chỉ ra cách đánh số
môi trường.
\begin{description}
   \item\exmacro{newtheorem\{vidu\}\{Ví dụ\}}\\
%%       The theorem set {\envfont foo} (whose name is \texttt{bar})
%%       uses its own counter.
	Định nghĩa môi trường |vidu|, với tên là |Ví dụ|
	(như vậy, bạn sẽ có |Ví dụ 1|, |Ví dụ 2|, \ldots). Môi trường
	này sử dụng bộ đếm riêng |vidu|, và bạn có thể thay đổi giá trị
	giá trị bộ đếm này, chẳng hạn |\setcounter{vidu}0|.
   \item\exmacro{newtheorem\{vidu2\}{[vidu]}\{Ví dụ khác\}}\\
%%       The theorem set {\envfont foo2} (printed name \texttt{bar2})
%%       uses the same counter as the theorem set \texttt{foo}.
	Định nghĩa môi trường |vidu2|, với tên là |Ví dụ khác|.
	Môi trường này sẽ sử dụng cùng bộ đếm của môi trường |vidu|
	trong ví dụ trước.
   \item\exmacro{newtheorem\{baitap\}\{Bài tập\}{[section]}}\\
%%       The theorem set {\envfont foo3} (printed name \texttt{bar}) is
%%       enumerated within the counter \texttt{section}, i.e.\ with every
%%       new |\section| the enumeration begins again with 1, and
%%       the enumeration is composed from the section-number and the
%%       theorem counter itself.
	Định nghĩa môi trường |baitap| (với tên là |Bài tập|), sử
	dụng bộ đếm thay đổi theo mục (|section|). Nếu bạn đang ở Mục số 5 chẳng hạn,
	bạn sẽ có |Bài tập 5.1|, |Bài tập 5.2|, \ldots Mỗi khi chuyển qua mục mới,
	bộ đếm sẽ được đặt về không, nghĩa là bạn sẽ có |Bài tập 6.1|, |Bài tập 6.2|, \ldots,
	|Bài tập 7.1|, |Bài tập 7.2|, \ldots
\end{description}

%% For every environment \meta{name}\ defined by |\newtheorem|,
%% \emph{two} enviroments \meta{name} and \meta{name|*|}\ are defined.
%% In the main document, they have exactly the same effect, but
%% the latter causes no entry in the respective list of theorems
%% (cf.\ |\section| and |\section*|), see also Section 
Khi gọi lệnh |\newtheorem| để tạo môi trường \meta{name}, thực ra sẽ có hai môi
trường được tạo ra, là \meta{name} và \meta{name|*|}. Điểm khác biệt duy nhất
giữa hai môi trường  này, cũng giống như sự khác biệt duy nhất giữa hai lệnh
|\section| và |\section*|, là môi trường \meta{name|*|} sẽ không đưa |THM|
vào trong danh sách liệt kê các |THM|. Trong các ví dụ ở trên, bạn sẽ có chẳng hạn
hai môi trường |baitap| và |baitap*|. Xem thêm Mục~\vref{sec:thmlists}.

\medskip
\DescribeMacro\renewtheorem\macro{renewtheorem}\\
%%
%% Theorem sets can be redefined by |\renewtheorem|, with the same arguments
%% as explained for |\newtheorem|. When redefining a theorem set, the 
%% counter is not re-initialized. 
Định nghĩa lại môi trường đã có. Cách dùng tương tự như của |\newtheorem|.
Bộ đếm sẽ được khởi tạo lại.

% =====================================================================

%% \subsection{Defining the Layout of Theorem Sets}
\subsection{Thay đổi kiểu dáng}

\label{sec:general}

%% For theorem-like environments, the user can set parameters
%% by setting several switches and then calling |\newtheorem|.
%% The layout of a theorem set is defined with the values of the switches
%% at the time |\newtheorem| is called.
Với các môi trường tựa định lý, bạn có thể thay đổi vài tham số (tuỳ chọn) trước
khi gọi lệnh |\newtheorem| để tinh chỉnh cách thể hiện môi trường
như ý bạn; các cài đặt nhờ tham số đó sẽ có tác dụng mỗi khi bạn
sử dụng môi trường.

% =====================================================================

%% \subsubsection{Common Parameters for all Theorem Sets}
\subsubsection{Các tham số chung}

\DescribeMacro\theorempreskipamount
\DescribeMacro\theorempostskipamount
\macro{theorempreskipamount} \macro{theorempreskipamount}\\ 
%% These additional parameters affect the vertical space around 
%% theorem environments:
%% |\theorempreskipamount| and |\theorempostskipamount| define,
%% respectively, the spacing before and after such an environment.
%% These parameters apply for all theorem sets and can be manipulated
%% with the ordinary length macros.  They are rubber lengths,
%% (`\textsf{skips}'), and therefore can contain \texttt{plus} and
%% \texttt{minus} parts.
Các tham số bổ sung này ảnh hưởng đến khoảng cách theo chiều đứng --
trên (|\theorempreskipamount|) và dưới (|\theorempostskipamount|) môi trường |THM|.
Hai tham số này ảnh hưởng đến mọi môi trường |THM| và có thể điều chỉnh
nhờ các lệnh thông thường điều khiển biến độ dài. Chúng là các chiều dài
dạng |rubber|, vì thế có thể chứa các phần với dấu cộng hoặc trừ.

%\subsubsection{Parameters for Individual Sets}
\subsubsection{Cho từng THM cụ thể}

%% The layout of individual theorem sets can be further determined
%% by switches controlling the appearance of the headers and the
%% header-body-layout:
Cách thể hiện của mỗi |THM| có thể tinh chỉnh nhờ các lệnh điều khiển sau đây.

\begin{description}
\item
\DescribeMacro\theoremstyle
  \exmacro{theoremstyle\{\meta{style}\}}\\
%%  The general structure of the 
%%  theorem layout is defined via its |\theoremstyle|. |\ntheorem|
%%  provides several predefined styles including those of
%%  Frank Mittelbach's |theorem.sty| 
	Xác định kiểu dáng của |THM|. Các kiểu được cung cấp với với |\ntheorem|
	bao gồm cả kiểu có trong gói |theorem.sty|. Xem liệt kê các kiểu
	ở Mục~\vref{sec:predefdstyles}.
%  Additional styles can be defined by |\newtheoremstyle| 
	Ở Mục~\vref{sec:newtheoremstyle} có nói về cách định nghĩa kiểu mới.
\item
\DescribeMacro\theoremheaderfont
  \exmacro{theoremheaderfont\{\meta{fontcmds}\}}\\
  %% The theorem header is set
  %%in the font specified by \meta{fontcmds}.
  Dùng \meta{fontcmds} để xác định |font| cho phần |header| của |THM|

%%   In contrast to |theorem.sty|, |\theoremheaderfont| can be set 
%%   individually for each environment type. 
	Không như |theorem.sty|, lệnh |\theoremheaderfont| cho phép đổi
	|font| cho từng kiểu |THM|.
\item
\DescribeMacro\theorembodyfont
  \exmacro{theorembodyfont\{\meta{fontcmds}\}}\\
  %The theorem body is set
  %%in the font specified by \meta{fontcmds}.
  Xác định |font| cho phần thân (nội dung) |THM|.
\item 
\DescribeMacro\theoremseparator
  \exmacro{theoremseparator\{\meta{sep}\}}\\
%%\meta{thing} %%separates the 
%%   header from the body of the theorem-environment. 
%%   E.g., \meta{thing} can be
%%   ``:'' or ``.''.
	Dùng \meta{sep} để ngăn cách phần |header| và phần thân của |THM|.
	Thường thì \meta{sep} là dấu hai chấm (|:|) hoặc chấm (|.|).
\item 
\DescribeMacro\theoremindent
%%   |\theoremindent{|\meta{dimen}|}| can be used to indent the theorem wrt.\
%%   the surrounding text.
	\exmacro{theoremindent\{\meta{dimen}\}}\\
	dùng để xác định |indent| (khoảng cách so với lề bên trái).

\DANGER
%% It's a `\textsf(dimen)', so the user shouldn't try to specify a
%% \texttt{plus} or \texttt{minus} part, cause this leads to an error.
	Ở đây, \meta{dimen} là kích thước thật sự. Nếu bạn dùng kiểu |rubber|
	với các dấu |plus| hoặc |minus| trong phần \meta{dimen}, bạn sẽ gặp lỗi.
\item
\DescribeMacro\theoremnumbering
  \exmacro{theoremnumbering\{\meta{style}\}}\\
%%   specifies the appearance of
%%   the numbering of the theorem set. Possible \meta{styles} are
   Kiểu đánh số cho |THM|. Các giá trị có thể là:
  |arabic| (default), |alph|, |Alph|, |roman|,
  |Roman|, |greek|, |Greek| và |fnsymbol|. 

%%   Clearly, if a theorem-environment uses the counter of another
%%   environment type, also the numbering style of that environment
%%   is used. 
	Rõ ràng, nếu môi trường |THM| sử dụng bộ đếm từ môi trường |XYZ| khác,
	thì kiểu đánh số của môi trường |THM| sẽ thừa hưởng từ |XYZ|.
\item
\DescribeMacro\theoremsymbol
  \exmacro{theoremsymbol\{\meta{thing}\}}\\
%%   This is only active if
%%   |ntheorem.sty| is loaded with option |[thmmarks]|. 
%%   \meta{thing} is set as an endmark at the end of every instance
%%   of the environment.
%%   If no symbol should appear, say |\theoremsymbol{}|.
	Lệnh này chỉ các tác dụng khi gói |ntheorem.sty| được nạp với tuỳ chọn |thmmarks|.
	Ở đây, \meta{thing} sẽ được dùng như |endmark|, tức dấu kết thúc cho |THM|.
	Nếu không muốn dùng |endmark| cho riêng môi trường |THM| nào, 
	dùng lệnh |\theoremsymbol{}|.
\end{description}

%% The flexibility provided by these command should relieve the
%% users from the ugly hacking in |\newtheorem| to fit most of  
%% the requirements stated by publishers or supervisors.
Nhờ các lệnh điều khiển trên, bạn có thể linh hoạt tạo ra các |THM| như ý,
mà không phải nhọc công và phải quan tâm nhiều đến yếu tố kỹ thuật.

\medskip
\DescribeMacro\theoremclass\exmacro{theoremclass\{\meta{theorem-type}\}}\\
%% With the command |\theoremclass{|\meta{theorem-type}|}| 
%% (where \meta{theorem-type} must be an already defined theorem type),
%% these parameters can be set to the values which were used when
%% |\newtheorem| was called for \meta{theorem-type}. 
Với lệnh này, \meta{theomrem-type} là kiểu |THM| đã được định nghĩa.
Với cách gọi này, các thiết lập về kiểu dáng ?????????

%% With |\theoremclass{LaTeX}|, the standard \LaTeX\ layout can be
%% chosen.
Với |\theoremclass{LaTeX}|, kiểu dáng chuẩn của \LaTeX{} cho các |THM|
sẽ được dùng.

%\subsubsection{Font Selection}
\subsubsection{Lựa chọn font}

%% From the document structuring point of view, theorem environments 
%% are regarded as special parts inside a document. Furthermore,
%% the theorem header is only a distinguished part of a theorem
%% environment.
Xét về mặt cấu trúc, mỗi |THM| là một phần đặc biệt
của tài liệu, trong đó, phần |header| được thiết kế để dễ dàng
phân biệt với phần còn lại của môi trường.
%% Thus, |\theoremheaderfont| inherits characteristics of 
%% |\theorembodyfont| which also inherits in characteristics of 
%% the font of the surrounding environment.
%% Thus, if for example |\theorembodyfont| is |\itshape| and 
%% |\theoremheaderfont| is |\bfseries| the font selected for the 
%% header will have the characteristics `bold extended italic'. 
%% If this is not desired, the corresponding property has to be
%% explicitly overwritten in |\theoremheaderfont|, e.g.
%% by |\theoremheaderfont{\normalfont\bfseries}|
Vì vậy, lệnh |\theoremheaderfont| thừa hưởng các đặc trưng của |\theorembodyfont|,
và đến lượt mình, |\theorembodyfont| thừa hưởng các thuộc tính của phần
tài liệu bên ngoài |THM| đang xét.

\medskip
Ví dụ:
nếu |\theorembodyfont| là |\itshape| và |\theoremheaderfont| là |\bfseries|,
thì phần |header| thực tế có kiểu \textbf{\textit{đậm và nghiêng}}.

\medskip
Nếu điều này làm bạn không vừa ý, cụ thể là bạn muốn phần |header|
chỉ được in đậm, có thể làm như sau:
\begin{example}
  \theoremheaderfont{\normalfont\bfseries}
\end{example}


%% \subsubsection{Predefined theorem styles}
\subsubsection{Các kiểu đã định nghĩa}

\label{sec:predefdstyles}

%% The following theorem styles are predefined, covering those
%% from |theorem.sty|:
Các kiểu dáng định lý sau đã có sẵn (như trong gói |theorem.sty|):
\begin{deflist}{nonumberbreak:}
   \item[plain]
%%       This theorem style emulates the original \LaTeX{} definition,
%%       except that additionally the parameters
%%       |\theorem...skipamount| are used.
		Như kiểu dáng của \LaTeX{} chuẩn,
		ngoại trừ tham số bổ sung |\theorem...skipamount|  được dùng.
   \item[break]
%%       In this style, the theorem header is followed by a line break.
		Phần |header| ngăn cách với phần thân |THM| bởi dòng mới\footnote{thực ra là một dấu ngắt dòng}.
   \item[change]
%%       Header number and text are interchanged, without a line break.
		Chỉ số và tên |THM| hoán đổi vị trí. Tuy nhiên, phần
		|header| sẽ theo sau ngay bởi phần thân |THM| (so sánh với kiểu
   \item[changebreak]
%%       Like \texttt{change}, but with a line break after the header.
		Là sự kết hợp hai kiểu |change| và |break|.
   \item[margin]
%%       The number is set in the left margin, without a line break.
		Chỉ số được bố trí ở lề trái, không ngắt dòng sau phần |header|.
   \item[marginbreak]
%%       Like \texttt{margin}, but with a line break after the header.
		Như |margin|, nhưng ngắt dòng sau phần |header|.
   \item[nonumberplain]
%%       Like \texttt{plain}, without number (e.g.\ for proofs).
		Như |plain|, nhưng không đánh số (dùng cho chứng minh,\ldots)
   \item[nonumberbreak]
%%       Like \texttt{break}, without number.
		Tổ hợp |break| và |nonumberplain|.
   \item[empty]
%%       No number, no name. Only the optional argument is typeset.
		Phần |header| chỉ gồm tên riêng (nếu có), còn chỉ số và tên của
		|THM| được bỏ qua.
\end{deflist}

%\subsubsection{Default Setting}
\subsubsection{Thiết lập mặc định}

%% If no option is given, i.e.\ |ntheorem.sty| is loaded by
%% |\usepackage{ntheorem.sty}|, the following default is set up:
Khi không có tùy chọn nào được chỉ ra, nghĩa là gói |ntheorem.sty|
được nạp đơn giản nhờ |\usepackage{ntheorem}|, các thiết lập sau
sẽ được dùng:
\begin{example} 
  \theoremstyle{plain}
  \theoremheaderfont{\normalfont\bfseries}
  \theorembodyfont{\itshape}
  \theoremseparator{}
  \theoremindent0cm
  \theoremnumbering{arabic}
  \theoremsymbol{}
\end{example}
%% Thus, by only saying |\newtheorem{...}{...}|, the user gets
%% the same layout as in standard \LaTeX.
Vì vậy, bằng cách dùng |\newtheorem{...}{...}|, bạn thu được
cách thể hiện giống hệt trong \LaTeX{} chuẩn.

\subsubsection{A Standard Set of Theorems}\label{sec:standard}

A standard configuration of theorem sets is provided within
the file |ntheorem.std|, which will be included by the option
|[standard]|. It uses the |amssymb| and |latexsym| (automatically
loaded) packages and defines the following sets:
\begin{nlist}{Definitions:}
 \item[Theorems:] % |Theorem|, |Lemma|, |Proposition|,
  |Corollary|, |Satz|, |Korollar|,
 \item[Definitions:] |Definition|,
 \item[Examples:] |Example|, |Beispiel|,
 \item[Remarks:] |Anmerkung|, |Bemerkung|, |Remark|,
 \item[Proofs:] |Proof| and |Beweis|.
\end{nlist}
These theorem sets seem to be the most frequently used environments 
in english and german
documents.

The layout is defined to be theoremstyle |plain|, bodyfont |\itshape|,
Headerfont |\bfseries|, and endmark (theoremsymbol)
|\ensuremath{_\Box}| for all theorem-like environments\footnote{Note, 
that mathmode is ensured for the symbol.}.
For the definition-, remark- and example-like sets,
the above setting is used, except bodyfont |\upshape|.
The proof-like sets are handled a bit differently. There, the layout 
is defined as theoremstyle |nonumberplain|, bodyfont |\upshape|,
headerfont |\scshape| and endmark |\ensuremath{_\blacksquare}|. 
For a more detailed information look at 
|ntheorem.std| or at the code-section.

\subsubsection{Customization and Local Settings}

Since the user should not change |ntheorem.std|,
we've added the possibility to use an own configuration-file.
If one places the file |ntheorem.cfg| in the path searched by
\TeX, this file is read automatically (if |[standard]|
is not given). The usage of |ntheorem.cfg| can be prevented by the
|[noconfig]| option. Thus, just
a copy of |ntheorem.std| to |ntheorem.cfg| must be made
which then can freely be modified by the user. Note, that if a 
configuration-file exists, this will always be used (I.e.\ with 
option |standard| and an existing configuration-file, the |.cfg| 
file will be used and the |.std| file won't.

\subsection{Generating Theoremlists}\label{sec:thmlists}

\DescribeMacro\listtheorems
Similar to the \LaTeX\ command |\listoffigures|, 
any theorem set defined with a |\newtheorem| statement
may be listed at any place in your document by
\begin{quote}
 |\listtheorems{|\meta{list}|}|
\end{quote}
The argument \meta{list} is a comma-separated list
of the theorem sets to be listed. 
For a theorem set \meta{name}, only the instances are listed 
which are instantiated by |\begin{|\meta{name}|}|. Those
instantiated by  |\begin{|\meta{name}|*}| are omitted
(cf.\ |\section| and |\section*|).

For example,
 |\listtheorems{Corollary,Lemma}|
leads to a list of all instances of one of the theorem sets 
``Corollary'' or ``Lemma''.
Note, that the set name given to the command is the first 
argument which is specified by |\newtheorem| which is also
the one to be used in |\begin{theorem} ... \end{theorem}|.

If |\listtheorems| is called for a set name which is not defined
via |\newtheorem|, the user is informed that a list is generated,
but there will be no typeset output at all.

\subsubsection{Defining the List Layout}

\DescribeMacro\theoremlisttype
Theoremlists can be formatted in different ways. Analogous to
theorem layout, there are several predefined types which can be
selected by
\begin{quote}
 |\theoremlisttype{|\meta{type}|}|
\end{quote}
The following four \meta{type}s are available (for examples, the
user is referred to section \ref{sec:examples}).
\begin{deflist}{allname}
 \item[all] List any theorem of the specified set by number,
   (optional) name and pagenumber. This one is also the
  default value.
 \item[allname] Like |all|, additionally with leading theoremname.
 \item[opt] Analogous to |all|, but only the theorems which have an 
    optional name are listed.
 \item[optname] Like |opt|, with leading theoremname.
\end{deflist}

\subsubsection{Writing Extra Stuff to Theorem File}

Similar to |\addcontentsline| and |\addtocontents|, 
additional entries to theoremlists are supported.
Since entries to theoremlists are a bit more intricate than
entries to the lists maintained by standard \LaTeX\, 
|\addcontentsline| and |\addtocontents| cannot be used in a
straightforward way\footnote{for a theorem, its number has
to be stored explicitly since different theorem sets can use
the same counter. Also, it is optional to reset the counter for
each section.}.

\DescribeMacro\addtheoremline
Analogous  to |\addcontentsline|, an extra entry for a theorem
list can be made by
\begin{quote}
 |\addtheoremline{|\meta{name}|}{|\meta{text}|}|
\end{quote}
where \meta{name} is the name of a valid theorem set and \meta{text}
is the text, which should appear in the list. For example, 
\begin{quote}
 |\addtheoremline{Example}{Extra Entry with number}|
\end{quote}
 \addtheoremline{Example}{Extra Entry with number}
generates an entry with the following characteristics:
\begin{itemize}
 \item The Label of the theorem ``Example'' is used.
 \item The current value of the counter for ``Example'' is used
 \item The current pagenumber is used.
 \item The specified text is the optional text for the theorem.
\end{itemize}
Thus, the above command has the same effect as it would be for
\begin{quote}
 |\begin{Example}[Extra Entry with number] \end{Example}|
\end{quote}
except, that there would be no output of the theorem, and the counter
isn't advanced.

\DescribeMacro{\addtheoremline*}
Alternatively you can use
\begin{quote}
 |\addtheoremline*{Example}{Extra Entry}|
\end{quote}
 \addtheoremline*{Example}{Extra Entry}
which is the same as above, except that the entry appears without
number.

\DescribeMacro\addtotheoremfile
Sometimes, e.g.\ for long lists, special control sequences 
(e.g.\ a pagebreak) or additional text should be inserted into a 
list. This is done by
\begin{quote}
 |\addtotheoremfile[|\meta{name}|]{|\meta{text}|}|
\end{quote}
where \meta{name} is the name of a theorem set and
\meta{text} is the text to be written into the theorem file.
If the optional argument \meta{name} is omitted, the given
text is inserted in every list, otherwise it is only inserted 
for the given theorem set.

\subsection{For Experts: Defining Layout Styles}
\subsubsection{Defining New Theorem Layouts}\label{sec:newtheoremstyle}

\DescribeMacro\newtheoremstyle
Additional layout styles for theorems can be defined by
\begin{quote}
 |\newtheoremstyle{|\meta{name}|}{|\meta{head}|}{|\meta{opt-head}|}|.
\end{quote} 
After this, |\theoremstyle{|\meta{name}|}| is a valid
|\theoremstyle|.
Here, \meta{head} has to be a statement using two arguments, 
|##1|, containing the keyword, and |##2|, containing the number. 
\meta{opt-head} has to be a statement using three arguments where
the additional argument |##3| contains the optional parameter.

Since \LaTeX\ implements theorem-like environments by |\trivlist|s,
both header declarations must be of the form
|\item[... \theorem@headerfont ...]...|, where
the dotted parts can be formulated by the user.
If there are some statements producing
output after the |\item[...]|, you have to care about implicit
spaces.

Because of the |@|, if |\newtheoremstyle| is used in a
|.tex| file, it has to be put between |\makeatletter| and
|\makeatother|.

For details, look at the code documentation or the
definitions of the predefined theoremstyles.

\DescribeMacro\renewtheoremstyle
Theorem styles can be redefined by |\renewtheoremstyle|, with the 
same arguments as explained for |\newtheoremstyle|.

\subsubsection{Defining New Theorem List Layouts}\label{sec:listtypes}

\DescribeMacro\newtheoremlisttype
Analogous, additional layouts for theorem lists can be defined by
\begin{quote}
 |\newtheoremlisttype{|\meta{name}|}{|\meta{start}|}{|\meta{line}%
 |}{|\meta{end}|}|.
\end{quote}
The first argument, \meta{name}, is the name of the listtype, 
which can the be used as a valid |\theoremlisttype|. 
\meta{start} is the sequence of commands to be executed at
the very beginning of the list. 
Corresponding, \meta{end} will be executed at the end of the list. 
These two are set to do nothing in the standard-types.
\meta{line} is the part to be called for every entry of the list. 
It has to be a statement using four arguments: |##1| will be 
replaced with the name of the theorem, |##2| with the number, 
|##3| with the theorem's optional text and |##4| with the pagenumber.

WARNING: Self-defined Layouts will break with the |hyperref|-package.

\DescribeMacro\renewtheoremlisttype
Theorem list types can be redefined by |\renewtheoremlisttype|, with 
the same arguments as explained for |\newtheoremlisttype|.

\subsection{Setting End Marks}

The automatic placement of endmarks is activated by calling
|ntheorem.sty| with the option |[thmmarks]|.
Since then, the endmarks are set automatically, there are only 
a few commands for dealing with very special situations.

\DescribeMacro\qed
\DescribeMacro\qedsymbol
If in a single environment, the user wants to replace the standard
endmark by some other, this can be done by saying |\qed|,
if |\qedsymbol| has been defined by |\qedsymbol{|\meta{something}|}|
(in option standard, |\qedsymbol| is defined to be the symbol
used for proofs, since a potential use of this features is to
close trivial corollaries without explicitly proving them).

Additionally, if in a single environment of a theorem set, that 
is defined without an endmark, the user wants to set an endmark, 
this is done with |\qedsymbol| and |\qed| as described above.
|\qedsymbol| can be redefined everywhere in the document.

\DescribeMacro\NoEndMark
\DescribeMacro\TheoremSymbol
On the other hand, if in some situation, the user decides to set 
the endmark manually (e.g.\ inside a figure or a minipage), the 
automatic handling can be turned off by |\NoEndMark| for the
current environment. 
Then -- assumed that he current environment is of type \meta{name},
the endmark can manually be set by just saying 
|\|\meta{name}|Symbol|.

Note that there must be no empty line in the input before the
|\end{theorem}|, since then, the end mark is ignored (cf.\
Theorem~\ref{ex-empty-line} in Section~\ref{sec:examples}).

\subsection{Extended Referencing Features}

The extended referencing features are activated by calling
|ntheorem.sty| with the option |[thref]|.

Often, when writing a paper, one changes propositions into
theorems, theorems into corollaries, lemmata into remarks
an so on. Then, it is necessary to adjust also the references,
i.e., from ``|see Proposition~\ref{completeness}|'' to
``|see Theorem~\ref{completeness}|''. For relieving the user 
from this burden, the type of the respective labeled entities
can be associated with the label itself:

\begin{quote}
|\label{|\meta{label}|}[|\meta{type}|]| 
\end{quote}
associates the type \meta{type} with \meta{label}. \\
This task is automated for theorem-like environments:

\begin{quote}
|\begin{Theorem}[|\meta{name}|]\label{|\meta{label}|}|
\end{quote}
is equivalent to
\begin{quote}
|\begin{Theorem}[|\meta{name}|]\label{|\meta{label}|}[Theorem]|
\end{quote}

The additional information is used by
\DescribeMacro\thref
\begin{quote}
|\thref{|\meta{label}|}|
\end{quote}
which outputs the respective environment-type \emph{and} the number,
e.g., ``Theorem~42''. Note that \LaTeX\ has to be run twice after
changing labels (similar to getting references OK; in the 
intermediate run, warnings about undefined reference types can
occur).

The |[thref]| option interferes with the |babel| package, thus in 
this case, |ntheorem| has to be loaded \emph{after} |babel|. It also
interferes with |amsmath|; see Section~\ref{sec:amslatex}.


\subsection{Miscellaneous}

Inside a theorem-like environment \meta{env}, the name given as optional 
argument is accessible by |\|\meta{env}|name|. 

\endinput
