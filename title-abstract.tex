% =====================================================================

\title{%An Extension of the \LaTeX-Theorem Evironment%
	%\thanks{This 
	%         file has version number \fileversion{}, last revised \filedate.}
	Mở rộng môi trường định lý của \LaTeX{}
}
\author{%
	Wolfgang May\thanks{\url{may@informatik.uni-freiburg.de}}
	\qquad 
	Andreas Schlechte\thanks{\url{ntheorem@andreas-schlechte.de}}\\[2mm]
	\emph{\underline{Biên dịch:}} kyanh\thanks{\url{kyanh@o2.pl}}%
}
\date{%
	Bản dịch số \textbf{\the\buildnum}\\[2mm]
	\fcolorbox{black}{red!20}{cho \textbf{ntheorem} bản \textbf{1.24} (2004/09/20)}}

\maketitle          

% =====================================================================

\begin{abstract}
%\noindent
%%  |ntheorem.sty| is a package for handling theorem-like environments.
%%  Aditionally to several features for defining the layout of
%%  theorem-like environments which can be regarded to be standard
%%  requirements for a theorem-package, it provides solutions for
%%  two related problems: placement of endmarks and generation of
%%  lists of theorem-like environments.
|ntherem.sty| là gói thể hiện các môi trường (tựa) định lý.
Bên cạnh các tính năng giúp thay đổi cách thể hiện môi trường (tựa) định lý,
gói còn giúp giải quyết vài vấn đề liên quan: đặt dấu kết thúc (|endmarks|),
tạo bảng liệt kê các định lý.

%%  In contrast to former approaches, it solves the problem of 
%%  setting endmarks of theorem-like environments (theorems, 
%%  definitions, examples, and proofs) \emph{automatically} at the 
%%  right positions, even if the environment ends with a |displaymath| 
%%  or (even nested) list environments, it also copes with the 
%%  |amsmath| package.
%%  This is done in the same manner as the handling of labels by
%%  using the |.aux| file.
\medskip
Trái với các cách tiếp cận trước đây, gói giải quyết vấn đề đặt dấu
kết thúc (|endmarks|) cho các môi trường tựa định lý (|theorem|,
|definition|, |example|, |proof|) một cách tự động, chính xác, ngay cả
đối với môi trường kết thúc bởi môi trường |displaymath| hoặc môi trường
danh sách (thậm chí các môi trường này có thể lồng nhau -- |nested|);
nhờ đó giải quyết được hoàn toàn các trục trặc khi dùng gói |amsmath|.
Nguyên lý làm việc của gói giống như cách \LaTeX{} điều khiển việc
đặt nhãn, bằng cách sử dụng các tập tin |.aux|.

%%  It also introduces the generation of lists of theorem-like 
%%  environments in the same manner as |listoffigures|. Additionally,
%%  more comfortable referencing is supported.
\medskip
Gói cung cấp lệnh để tạo danh sách các môi trường tựa định lý,
tương tự như khi liệt kê các hình vẽ bằng |\listoffigures|.
% Ngoài ra, gói hỗ trợ 

%%  After running \LaTeX\ several times (depending on the complexity
%%  of references, in general, three runs are sufficient), the endmarks 
%%  are set correctly, and theoremlists are generated.
\medskip
Sau khi biên dịch tài liệu vài lần (số lần tuỳ thuộc vào sự phức tạp
của các tham khảo chéo; thường thì ba lần là đủ), các dấu kết thúc (|endmarks|)
sẽ được đặt đúng chỗ, và danh sách các định lý sẽ được tạo ra.

%%  Since |ntheorem.sty| uses the standard \LaTeX\ |\newtheorem|
%%  command, existing documents can be switched to
%%  |ntheorem.sty| without having to change the |.tex| file.
%%  Also, it is compatible with \LaTeX\ files using |theorem.sty| 
%%  written by Frank Mittelbach.
\medskip
Do gói |ntheorem.sty| sử dụng lệnh |\newtheorem| của \LaTeX{} chuẩn,
các tài liệu cũ có thể chuyển qua dùng gói mà không cần thay đổi nội
dung. Ngoài ra, gói còn tương thích với các tài liệu dùng gói |theorem.sty|
của Frank Mitterbach.
\end{abstract}

\newpage

\tableofcontents

\endinput
